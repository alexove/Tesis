%---------------------------------------------------------------------
%
%                      agradecimientos.tex
%
%---------------------------------------------------------------------
%
% agradecimientos.tex
% Copyright 2009 Marco Antonio Gomez-Martin, Pedro Pablo Gomez-Martin
%
% This file belongs to the TeXiS manual, a LaTeX template for writting
% Thesis and other documents. The complete last TeXiS package can
% be obtained from http://gaia.fdi.ucm.es/projects/texis/
%
% Although the TeXiS template itself is distributed under the
% conditions of the LaTeX Project Public License
% (http://www.latex-project.org/lppl.txt), the manual content
% uses the CC-BY-SA license that stays that you are free:
%
%    - to share & to copy, distribute and transmit the work
%    - to remix and to adapt the work
%
% under the following conditions:
%
%    - Attribution: you must attribute the work in the manner
%      specified by the author or licensor (but not in any way that
%      suggests that they endorse you or your use of the work).
%    - Share Alike: if you alter, transform, or build upon this
%      work, you may distribute the resulting work only under the
%      same, similar or a compatible license.
%
% The complete license is available in
% http://creativecommons.org/licenses/by-sa/3.0/legalcode
%
%---------------------------------------------------------------------
%
% Contiene la pXgina de agradecimientos.
%
% Se crea como un capXtulo sin numeraciXn.
%
%---------------------------------------------------------------------

\chapter{Agradecimientos}

\cabeceraEspecial{Agradecimientos}

\begin{FraseCelebre}
\begin{Frase}
A todos los que la presente vieren y entendieren.
\end{Frase}
\begin{Fuente}
Inicio de las Leyes Orgánicas. Juan Carlos I
\end{Fuente}
\end{FraseCelebre}

Groucho Marx decía que encontraba a la televisión muy educativa porque
cada vez que alguien la encendía, él se iba a otra habitación a leer
un libro. Utilizando un esquema similar, nosotros queremos agradecer
al Word de Microsoft el habernos forzado a utilizar \LaTeX. Cualquiera
que haya intentado escribir un documento de más de 150 páginas con
esta aplicación entenderá a qué nos referimos. Y lo decimos porque
nuestra andadura con \LaTeX\ comenzó, precisamente, después de
escribir un documento de algo más de 200 páginas. Una vez terminado
decidimos que nunca más pasaríamos por ahí. Y entonces caímos en
\LaTeX.

Es muy posible que hubXeramos llegado al mismo sitio de todas formas,
ya que en el mundo acadXmico a la hora de escribir artXculos y
contribuciones a congresos lo mXs extendido es \LaTeX. Sin embargo,
tambiXn es cierto que cuando intentas escribir un documento grande
en \LaTeX\ por tu cuenta y riesgo sin un enlace del tipo ``\emph{Author
  instructions}'', se hace cuesta arriba, pues uno no sabe por donde
empezar.

Y ahX es donde debemos agradecer tanto a Pablo GervXs como a Miguel
Palomino su ayuda. El primero nos ofreciX el cXdigo fuente de una
programaciXn docente que habXa hecho unos aXos atrXs y que nos sirviX
de inspiraciXn (por ejemplo, el fichero \texttt{guionado.tex} de
\texis\ tiene una estructura casi exacta a la suya e incluso puede
que el nombre sea el mismo). El segundo nos dejX husmear en el cXdigo
fuente de su propia tesis donde, ademXs de otras cosas mXs
interesantes pero menos curiosas, descubrimos que aXn hay gente que
escribe los acentos espaXoles con el \verb+\'{\i}+.

No podemos tampoco olvidar a los numerosos autores de los libros y
tutoriales de \LaTeX\ que no sXlo permiten descargar esos manuales sin
coste adicional, sino que tambiXn dejan disponible el cXdigo fuente.
Estamos pensando en Tobias Oetiker, Hubert Partl, Irene Hyna y
Elisabeth Schlegl, autores del famoso ``The Not So Short Introduction
to \LaTeXe'' y en TomXs Bautista, autor de la traducciXn al espaXol. De
ellos es, entre otras muchas cosas, el entorno \texttt{example}
utilizado en algunos momentos en este manual.

TambiXn estamos en deuda con JoaquXn Ataz LXpez, autor del libro
``CreaciXn de ficheros \LaTeX\ con {GNU} Emacs''. Gracias a Xl dejamos
de lado a WinEdt y a Kile, los editores que por entonces utilizXbamos
en entornos Windows y Linux respectivamente, y nos pasamos a emacs. El
tiempo de escritura que nos ahorramos por no mover las manos del
teclado para desplazar el cursor o por no tener que escribir
\verb+\emph+ una y otra vez se lo debemos a Xl; nuestro ocio y vida
social se lo agradecen.

Por Xltimo, gracias a toda esa gente creadora de manuales, tutoriales,
documentaciXn de paquetes o respuestas en foros que hemos utilizado y
seguiremos utilizando en nuestro quehacer como usuarios de
\LaTeX. SabXis un montXn.

Y para terminar, a Donal Knuth, Leslie Lamport y todos los que hacen y
han hecho posible que hoy puedas estar leyendo estas lXneas.

\endinput
% Variable local para emacs, para  que encuentre el fichero maestro de
% compilaciXn y funcionen mejor algunas teclas rXpidas de AucTeX
%%%
%%% Local Variables:
%%% mode: latex
%%% TeX-master: "../Tesis.tex"
%%% End:
