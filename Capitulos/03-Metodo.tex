%---------------------------------------------------------------------
%                          Capítulo 3
%---------------------------------------------------------------------

\chapter{Método}
\section{Alcance del Estudio}
En palabras de \cite{sampieri}, el alcance del estudio es importante por que de
ésta dependera la estrategia de la investigación y como consecuencia el diseño,
procedimientos y otros componentes del proceso serán distintos de acuerdo al alcance
planteado.

De acuerdo a \cite{sampieri} existen 4 tipos de alcance, los cuales son:
\begin{itemize}
    \item \textbf{Exploratorios}, utilizados en problemas poco estudiados, indagan
          desde una perspectiva innovadora, ayudan a identificar conceptos promisorios,
          preparan el terreno para nuevos estudios.
    \item \textbf{Descriptivos}, consideran al fenómeno estudiando y sus componentes,
          miden conceptos y definen variables.
    \item \textbf{Correlacionales}, asocian conceptos o variables, permiten predicciones,
          cuantifican relaciones entre conceptos o variables.
    \item \textbf{Explicativos}, determinan las causas de los fénomenos, generan un
          sentido de entendimiento, son sumamente estructurados.
\end{itemize}

El estudio que se realizará en el Centro de Desarrollo Empresarial del Cusco será
de \textbf{alcance explicativo}, debido a que se pretende determinar si el uso
de herramientas de la computación en la nube, especificamente el modelo de
\emph{Software como Servicio}, ayuda a mejorar la gestión empresarial de las empresas
beneficiarias del Centro de Desarrollo Empresarial del Cusco.

\section{Diseño de investigación}
En palabras de \cite{sampieri}, la gestación del diseño del estudio representa
el punto donde se conectan las etapas conceptuales del proceso de investigación
como el planteamiento del problema, el desarrollo de la perspectiva teórica y
la hipótesis con las fases subsecuentes cuyo carácter es más operativo.

Además \cite{sampieri} agrega que el propósito del diseño de investigación
es:
\begin{itemize}
    \item Responder preguntas de investigación.
    \item Cumplir objetivos del estudio.
    \item Someter hipótesis a prueba.
\end{itemize}

Respecto a los tipos de diseños disponibles, \cite{sampieri} indica que existen
dos tipos principales: investigación experimental e investigación no experimental.
A su vez, la primera puede dividirse de acuerdo con las clásicas categorías de
Campbell y Stanley (1966) en: preexperimentos, experimentos ``puro'' y cuasiexperimentos.
La investigación no experimental la subdividimos en diseños transversales y diseños
longitudinales.

\cite{sampieri} señala respecto a los \emph{preexperimentos} que se denominan de
esa forma debido a que su grado de control es mínimo. Se caracteriza por investigar
un solo grupo cuyo grado de control es mínimo. Generalmente es útil como primer
acercamiento al problema de investigación en la realidad.

Debido a que el enfoque para este trabajo es cuantitativo, se deberá realizar
una \textbf{investigación experimental} siendo el tipo específico la \textbf{preexperimental}.

\section{Población}
% TODO: Definir tacitamente que es el CDE (definicion formal o legal) y complementar
% esta sección
Como se menciono en secciones anteriores, el trabajo de investigación se realizará
con la empresas que son beneficiarias del Centro de Desarrollo Empresarial del Cusco,
constituido en la ciudad del Cusco por convenio entre el CEC Guaman Poma de Ayala
y el Ministerio de la Producción.

En este caso no se tomarán muestras sino que se considerará la población de
empresas en todo su conjunto debido a que la cantidad es pequeña, aproximadamente
30 empresas.

\section{Técnicas e instrumentos de recolección de datos}
Según \cite{robledo} en función de las distintas técnicas que se aplican para
la obtención de los datos o evidencias, se distinguen tres áreas las cuales son:
\begin{itemize}
    \item Investigación documental.
    \item Investigación de campo.
    \item Investigación de laboratorio,.
\end{itemize}

La investigación documental consiste en la recolección de información ya existente
que este debidamente documentada, según \cite{robledo} agrega que en este tipo
de investigación se distinguen distintos instrumentos, haciendo hincapié en las
\emph{fichas}, las cuales \citep{robledo} las clasifica en:
\begin{itemize}
    \item Fichas bibliograficas.
    \item Fichas de trabajo.
\end{itemize}

Respecto a las fichas de trabajo \cite{robledo} indica que es el instrumento de
trabajo intelectual que se utiliza para recabar, registrar, clasificar y
manejar los datos relacionados con un problema de investigación.

Bajo este entender, las técnicas que se aplicarán para este trabajo serán:
la \textbf{investigación documental} e \textbf{investigación de campo} utilizando
como instrumentos especificos las \textbf{fichas de trabajo} y \textbf{encuestas}.

%\section{Validez y confiabilidad de instrumentos}

\section{Plan de Análisis de datos}
Para el análisis de datos se utilizará la \textbf{Prueba T de Student, Prueba
T-Student o Test-T}, ya que es la que más se adecua a nuestro escenario.
