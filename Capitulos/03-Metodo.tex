%---------------------------------------------------------------------
%                          Cap\'itulo 3
%---------------------------------------------------------------------

\chapter{M\'etodo}
\section{Alcance del Estudio}
En palabras de \cite{sampieri}, el alcance del estudio es importante por que de
\'esta dependera la estrategia de la investigaci\'on y como consecuencia el dise\~no,
procedimientos y otros componentes del proceso ser\'an distintos de acuerdo al alcance
planteado.

De acuerdo a \cite{sampieri} existen 4 tipos de alcance, los cuales son:
\begin{itemize}
    \item \textbf{Exploratorios}, utilizados en problemas poco estudiados, indagan
          desde una perspectiva innovadora, ayudan a identificar conceptos promisorios,
          preparan el terreno para nuevos estudios.
    \item \textbf{Descriptivos}, consideran al fen\'omeno estudiando y sus componentes,
          miden conceptos y definen variables.
    \item \textbf{Correlacionales}, asocian conceptos o variables, permiten predicciones,
          cuantifican relaciones entre conceptos o variables.
    \item \textbf{Explicativos}, determinan las causas de los f\'enomenos, generan un
          sentido de entendimiento, son sumamente estructurados.
\end{itemize}

El estudio que se realizar\'a en el Centro de Desarrollo Empresarial del Cusco ser\'a
de \textbf{alcance explicativo}, debido a que se pretende determinar si el uso
de herramientas de la computaci\'on en la nube, especificamente el modelo de
\emph{Software como Servicio}, ayuda a mejorar la gesti\'on empresarial de las empresas
beneficiarias del Centro de Desarrollo Empresarial del Cusco.

\section{Dise\~no de investigaci\'on}
En palabras de \cite{sampieri}, la gestaci\'on del dise\~no del estudio representa
el punto donde se conectan las etapas conceptuales del proceso de investigaci\'on
como el planteamiento del problema, el desarrollo de la perspectiva te\'orica y
la hip\'otesis con las fases subsecuentes cuyo car\'acter es m\'as operativo.

Adem\'as \cite{sampieri} agrega que el prop\'osito del dise\~no de investigaci\'on
es:
\begin{itemize}
    \item Responder preguntas de investigaci\'on.
    \item Cumplir objetivos del estudio.
    \item Someter hip\'otesis a prueba.
\end{itemize}

Respecto a los tipos de dise\~nos disponibles, \cite{sampieri} indica que existen
dos tipos principales: investigaci\'on experimental e investigaci\'on no experimental.
A su vez, la primera puede dividirse de acuerdo con las cl\'asicas categor\'ias de
Campbell y Stanley (1966) en: preexperimentos, experimentos ``puro'' y cuasiexperimentos.
La investigaci\'on no experimental la subdividimos en dise\~nos transversales y dise\~nos
longitudinales.

\cite{sampieri} se\~nala respecto a los \emph{preexperimentos} que se denominan de
esa forma debido a que su grado de control es m\'inimo. Se caracteriza por investigar
un solo grupo cuyo grado de control es m\'inimo. Generalmente es \'util como primer
acercamiento al problema de investigaci\'on en la realidad.

Debido a que el enfoque para este trabajo es cuantitativo, se deber\'a realizar
una \textbf{investigaci\'on experimental} siendo el tipo espec\'ifico la \textbf{preexperimental}.

\section{Poblaci\'on}
% TODO: Definir tacitamente que es el CDE (definicion formal o legal) y complementar
% esta sección
Como se menciono en secciones anteriores, el trabajo de investigaci\'on se realizar\'a
con la empresas que son beneficiarias del Centro de Desarrollo Empresarial del Cusco,
constituido en la ciudad del Cusco por convenio entre el CEC Guaman Poma de Ayala
y el Ministerio de la Producci\'on.

En este caso no se tomar\'an muestras sino que se considerar\'a la poblaci\'on de
empresas en todo su conjunto debido a que la cantidad es peque\~na, aproximadamente
30 empresas.

\section{T\'ecnicas e instrumentos de recolecci\'on de datos}
Seg\'un \cite{robledo} en funci\'on de las distintas t\'ecnicas que se aplican para
la obtenci\'on de los datos o evidencias, se distinguen tres \'areas las cuales son:
\begin{itemize}
    \item Investigaci\'on documental.
    \item Investigaci\'on de campo.
    \item Investigaci\'on de laboratorio,.
\end{itemize}

La investigaci\'on documental consiste en la recolecci\'on de informaci\'on ya existente
que este debidamente documentada, seg\'un \cite{robledo} agrega que en este tipo
de investigaci\'on se distinguen distintos instrumentos, haciendo hincapi\'e en las
\emph{fichas}, las cuales \citep{robledo} las clasifica en:
\begin{itemize}
    \item Fichas bibliograficas.
    \item Fichas de trabajo.
\end{itemize}

Respecto a las fichas de trabajo \cite{robledo} indica que es el instrumento de
trabajo intelectual que se utiliza para recabar, registrar, clasificar y
manejar los datos relacionados con un problema de investigaci\'on.

Bajo este entender, las t\'ecnicas que se aplicar\'an para este trabajo ser\'an:
la \textbf{investigaci\'on documental} e \textbf{investigaci\'on de campo} utilizando
como instrumentos especificos las \textbf{fichas de trabajo} y \textbf{encuestas}.

%\section{Validez y confiabilidad de instrumentos}

\section{Plan de An\'alisis de datos}
Para el an\'alisis de datos se utilizar\'a la \textbf{Prueba T de Student, Prueba
T-Student o Test-T}, ya que es la que m\'as se adecua a nuestro escenario.
