%---------------------------------------------------------------------
%             Capítulo 1
%---------------------------------------------------------------------

\chapter{Introducción}

%---------------------------------------------------------------------
%            Planteamiento del Problema
%---------------------------------------------------------------------
\section{Planteamiento del problema}
Las micro, pequeñas y medianas empresas, en adelante MiPYMEs, ejercen un papel
esencial en la economia peruana porque son una de las principales fuentes de empleo
del país, disminuyen la pobreza y dan un aporte importante al PBI.

Antes de iniciar con nuestros análisis debemos de tener en cuenta que las MiPYMEs
según el \cite{produce2} están divididas en 3 estratos de acuerdo
a la Ley Nro. 30056 que los clasifica de acuerdo al promedio de ventas anuales
(en UIT) en:
\begin{itemize}
  \item Micro empresa, hasta 150 UIT.
  \item Pequeña empresa, de 150 a 1700 UIT.
  \item Mediana empresa, de 1700 a 2300 UIT.
\end{itemize}

Además debemos tener en cuenta que las MiPYMEs son la forma de empresa predominante
en el país que según el \cite{produce} la mayoría de las empresas son
microempresas (95,0\%). El estrato de las PYME (pequeñas y medianas empresas) presenta una baja participación,
con 4,3\% de pequeñas empresas y 0,2\% de medianas empresas. Siendo las regiones
con mayor número de MiPYME son Lima, Arequipa, La Libertad, Cusco y Piura, las que
en conjunto constituyen el 66,3\% del total de las MiPYME peruanas.

Tambien es importante saber cuales son los rubros en los que se concentra la mayor cantidad de empresas,
al respecto \cite{produce} señala que los rubros dominantes son el comercio (44,3\%) y servicios (41,1\%).

Claramente se denota la importancia de las MiPYME en la economía nacional debido a su aporte a la
producción nacional que según \cite{arbulu} fue de 42\% el año 2005 y por
su potencial de absorción de empleo que fue cerca del 88\%.

Además \cite{arbulu} agrega que tal es la importancia de las MiPYMES que éstas aportarón
49\% al PBI el año 2005, mientras el que \citep{produce} indica que el aporte
de las MiPYMES al PBI fue de 40\% el año 2013 lo que significa un descenso significativo,
sin embargo ambos casos porcentajes no dejan de ser importantes.

Respecto a las MiPYMES y el empleo \cite{produce2} muestra que 10 de cada 100 personas
de la Población Economicamente Activa (PEA) son conductoras de una MiPYME formal.
Además agrega que éstas generán el 60\% de la PEA ocupada. El 83,5\% de las
MiPYME formales tienen hasta cinco trabajadores, esta estructura no presenta cambios
sustanciales respecto a años anteriores y es explicado por el gran número de
MiPYME que se constituyen como personas naturales y presentan bajos niveles de
ventas. Sin embargo, esta proporción varía de acuerdo con el estrato empresarial:
el 86,7\% de las microempresas, 14,4\% de las pequeñas empresas y 4,3\% de las
medianas empresas tienen como máximo cinco trabajadores.

Gracias a los parrafos anteriores, se puede notar la importancia de las MiPYMEs en
la economia nacional tanto en la parte productiva, generación de divisas y empleo.
Sin embargo, existen problemas que no permiten el desarrollo de este tipo de empresas
y que el país no se beneficie adecuadamente del trabajo de éstas. Uno de los problemas que aqueja a
las MiPYMEs es la informalidad. \cite{loayza} precisa que la informalidad está
constituido por el conjunto de empresas, trabajadores y actividades que operan
fuera de los marcos legales y normativos que rigen la actividad económica. Por
lo tanto, pertenecer al sector informal supone estar al margen de las cargas
tributarias y normas legales, pero también implica no contar con la protección
y los servicios que el Estado puede ofrecer.

Según \cite{arbulu} el desarrollo de la MiPYME y del sector informal urbano
en el Perú ha sido un fenómeno característico de las últimas dos
décadas, debido al acelerado proceso de migración y urbanización que
sufrieron muchas ciudades, la aparición del autoempleo y de una gran cantidad
de unidades económicas de pequeña escala, frente a las limitadas fuentes
de empleo asalariada y formal para el conjunto de integrantes de la PEA (Población
Economicamente Activa).

\cite{loayza} señala desde un punto de vista conceptual que la informalidad surge
cuando los costos de circunscribirse al marco legal y normativo de un país son
superiores a los beneficios que ello conlleva. La formalidad involucra costos
tanto en términos de ingresar a este sector - largos,complejos y costosos procesos
de inscripción y registro - como en términos de permanecer dentro del mismo -pago
de impuestos, cumplir las normas referidas a beneficios laborales
y remuneraciones, manejo ambiental, salud, entre otros. A lo que \citep{penaranda}
agrega que existen una diversidad de enfoques que explican la presencia de la
informalidad en un sistema económico, como las barreras burocráticas y los sobrecostos
laborales y tributarios, así como también los distintos mecanismos a través de
los cuales esta afecta la productividad y el potencial de crecimiento de una
economía. En el accionar de una economía informal se encuentran empresas que
buscan eludir el control del Estado, manteniendo un tamaño inferior al óptimo
para gozar de beneficios tributarios o laborales, empleando mecanismos irregulares
para la compra de bienes y servicios e incluso destinando recursos financieros
para encubrir actividades ilegales.

Respecto al número de MiPYMEs informales \cite{produce2} señala que existe una
amplia heterogeneidad en las dimensiones de la informalidad, lo que dificulta
encontrar el número de empresas formales e informales de una manera única y precisa.
Al respecto \citep{produce2} cita a (Diaz, 2014) que sintetiza en dos dimensiones
la informalidad: informalidad laboral e informalidad tributaria. En el caso de la
primera se distingue varios criterios (rasgos) para identificar las obligaciones
propias de una relación laboral como el acceso a un seguro de salud, una pensión
de jubilación, gratificaciones y a un contrato de trabajo. En el caso de la segunda,
tambián distingue criterios, como la tenencia de RUC de la empresa, si esta tiene
un sistema de contabilidad, y si se encuentra registrada como persona jurídica.

Pese a las dificultades que representa determinar el numero exacto de empresas
informales \cite{produce2} estimó que el año 2010 existia un 30,4\% de empresas
formales frente a un 69,6\% de empresas informales (estimado), sin embargo las
cifras el año 2014 mostraron que el porcentaje de empresas formales aumento a 43,9\%
y el porcentaje de empresas informales descendio a 56,1\% que aún sigue siendo
un numero bastante alto de informalidad.

Respecto a la informalidad \cite{inei2} indica que el mayor número de unidades
productivas informales se concentra en la actividad Agropecuaria (33,8\%), le
siguen las actividades de Comercio (23,9\%), Transportes (12,2\%) y Otros servicios (10,9\%).

\cite{perucamaras} expone un análisis sobre la informalidad en la Macro Región
Sur de nuestro país donde el 79,1\% de la población ocupada labora en la informalidad,
mientras que solo el 20,9\% son trabajadores formales, agregando además que
2'013,914 personas en esta parte del país cuentan con empleos que no estan sujetos
a la legislación laboral nacional o que no pertenecen al sector formal de la
economia. Respecto a la actividad economica, se observa que en los sectores agropecuario
y pesquero la tasa de informalidad es de 99,2\%. Estas actividades concentran al
32,2\% de la Población Economicamente Activa (PEA) ocupada de esta macro región.

En la región del Cusco, según \cite{perucamaras}, el 83\% de la población
ocupada labora en la informalidad, mientras que solo el 17\% son trabajadores formales.

\cite{perucamaras} también agrega que según la actividad económica, la mayor
fuerza laboral está concentrada en los sectores agropecuario y pesquero (42\% de
la PEA ocupada), cuya tasa de informalidad es de 100\%. Los sectores de comercio
y servicios registran tasas de informalidad de 83,6\% y 78,2\%, respectivamente.
Dichas actividades representan el 15,5\% y 21,1\% de la PEA ocupada, respectivamente.
En esta región el 43,1\% de la población ocupada trabaja como independiente no
profesional y la informalidad alcanza al 91,8\%. Mientras que el 21,4\% se desempeña
como trabajador familiar no remunerado, actividad 100\% informal, y el 16\% labora en
microempresas, siendo el 83,1\% informal.

Respecto al uso de las Tecnologías de Información y Comunicaciones (TIC) en las
MiPYMES del Perú, \cite{inei1} señala que el 67,0\% de las Micro y Pequeña Empresas
manifestaron contar por lo menos con una computadora de escritorio, el 48,8\% un
equipo multifuncional, el 35,1\% una computadora portátil, el 27,3\% una impresora,
el 23,5\% un teléfono con acceso a internet (Smartphone), el 8,6\% un escáner y
el 6,0\% una fotocopiadora. A  nivel  de  ciudad,  se  observa  que  en la ciudad
del Cusco el 50\% de MiPYMEs cuentan con una computadora de escritorio, 31,7\% con
una computadora portatil, 36,6\% con un equipo multifuncional, 18,3\% con una impresora,
6,1\% con un scáner, 1,2\% con una fotocopiadora, 12,2\% con un smart phone con acceso
Internet.

Respecto al acceso a los servicios informáticos que disponen las computadoras de
las MiPYMEs \cite{inei1} indica que El 90,9\% declararon contar con servicios de
internet, el 0,9\% declaró tener intranet y el 9,0\% declaró no contar con
servicios informáticos. Las ciudades de Cusco con 37,3\% y Juliaca con 36,6\%,
presentan los más altos porcentajes de empresas que no cuentan con servicios
informáticos en sus computadoras.

Otro aspecto importante de las MiPYMEs y su relación con las TIC es el uso de los
sistemas de de gestión, al respecto \citep{inei1} señala que el 11,7\% contaban
con sistemas contable - tributario y el 7,1\% de ventas. En orden de importancia
le siguen los sistemas de producción con 5,0\%, personal con 2,7\% y soporte informático
con 2,3\%. Es oportuno resaltar que el 52,6\% de Micro y Pequeña Empresa no contaban
con ningún tipo de sistema de gestión. En la ciudad del Cusco los resultados
muestran que el 6,1\% de MiPYMEs cuentan con un sistema de información contable-tributario,
2,4\% con sistema de ventas, 2,4\% sistema de personal, 1,2\% sistema de finanzas,
1,2\% sistema logístico, 2,4\% sistemas de producción, 2,4\% sistemas de soporte
informático y un preocupante 52,4\% no tiene ningun sistema de información ni de
gestión.

A los datos proporcionados anteriormente respecto al uso de las TIC en las MiPYMEs
\cite{ipsos} proporciona datos adicionales como la gestión de las TIC dentro
de la empresa y al uso de la computación en la nube en la gestión del negocio.

Con respecto a la gestión de las TIC en las mypes \citep{ipsos} muestra que un
45\% de las empresas delega el manejo de las TIC a los empleados que los utilizan
(autoservicio), el 14\% lo administra un proveedor externo de TI, un 13\% son administrados
por un departamento interno de TI y un 28\% no maneja recursos de TI. Respecto a
esto podemos decir que el segundo y tercer caso son una forma adecuada de manejar
los recursos de TI, sin embargo estan en un segundo plano.

En adición \cite{ipsos} muestra que en un 71\% de MiPYMEs el dueño, fundador o director
ejecutivo es el responsable de decidir la compra y uso de TI, el 7\% lo decide
gerente especializado de TI, director de tecnología (CTO) o un director de información
(CIO), en un 3\% de casos lo decide un comité de tecnología, en un 2\% lo decide
cada empleado, en un 16\% lo deciden otros y 1\% no sabe. Respecto a esto debemos
decir que el segundo y tercer caso son los más adecuados y el primero puede ser la
causa de la mayoria de problemas de las MiPYMEs porque el criterio de ahorro de dinero
se anteponen a la conveniencia de dicha tecnología.

Con respecto al nivel de inversión en TIC \cite{ipsos} señala que un 44\% de MiPYMEs
gasta muy poco en las TIC, seguido de un 34\% gasta solo en lo necesario en las TIC,
un 10\% gasta un poco menos de lo necesario, otro 10\% gasta algo más de lo necesario y
1\% gasta demasiado en TIC. Obviamente lo preocupante en este resultado es que la
mayoria de empresas no considera importante a las TIC para la gestión de su empresa.

Por otro lado \cite{ipsos} señala que ante la frase "Entiendo completamente el
término computación en la nube para los negocios", un 28\% esta muy de acuerdo, un 32\%
esta en algo de acuerdo, 12\% es neutral (ni de acuerdo ni en desacuerdo), 12\%
algo en desacuerdo, 11\% muy en desacuerdo y 5\% no sabe. Esto indica que un grupo
considerable de entrevistados conoce la computación en la nube lo que aporta a que
dentro de un tiempo puedan adoptarla.

Ante la propuesta de que las empresas destinen mayor parte de su presupuesto a
soluciones de computación en la nube \cite{ipsos} indica que un 18\% de MiPYMEs
considera que sera muy probable, 46\% probable, 20\% no muy probable, 13\% nada
probable y 3\% no sabe. Esto indica que estas empresas requeriran más soluciones
de este tipo en el futuro.

En resumen, vemos en los parrafos precedentes que se da de manifiesto la importancia
de las MiPYMEs, debido a que generan divisas, aportan al PBI del país y generan
un porcentaje importante del empleo a nivel nacional. Sin embargo, tambien se da
de manifiesto que uno de los principales problemas de estas empresas es la informalidad
que impacta en la recaudación de impuestos y causa la informalidad en el empleo
lo que conlleva a que los colaboradores (empleados) no gocen de los beneficios de la formalidad.
Por otro lado, a pesar de que la mayoria de MiPYMEs cuenta con distintos equipos
informáticos y acceso a Internet, esto no constituye una ayuda o ventaja competitiva
lo que queda de manifiesto en la ausencia del uso de sistemas de información en
apoyo a la gestión.

Todos estos hechos traen como consecuencia que las MiPYMEs se verán limitadas
significativamente en su crecimiento ya que no aprovechan las TIC adecuadamente
que les ayudaria a simplificar las labores de gestión (labores financieras, contables,
recursos humanos, etc) y tambien tener acceso a los mercados nacionales e internacionales.

Es necesario que las empresas tengan acceso a las TIC para que puedan mejorar la
gestión sus recursos y planteen tacticas de crecimiento reales. La computación
en la nube ofrece la posibilidad de tener acceso obicuo (en todo lugar) a bajo
costo a las TIC, en especial a los sistemas de gestión para la empresa.

%---------------------------------------------------------------------
%            Formulación del problema
%---------------------------------------------------------------------
\section{Formulación del problema}
\subsection{Problema General}
?`Cual es el efecto del uso de las herramientas de la computación en la nube
en la gestión de las MiPYME pertenecientes al Centro de Desarrollo Empresarial
del Cusco?
\subsection{Problemas Espec'ificos}
\begin{enumerate}[a.]
\item ?`Cuál es el nivel de gestión de las MiPYMES pertenecientes al Centro
de Desarrollo Empresarial del Cusco antes del uso de herramientas de la computación
en la nube?
\item ?`Cuál es el nivel de gestión de las MiPYMES pertenecientes al Centro
de Desarrollo Empresarial del Cusco después del uso de herramientas de la computación
en la nube?
\end{enumerate}

%\subsubsection{Alcance Descriptivo}
%\subsubsection{Alcance Correlacional}
%\subsubsection{Alcance Explicativo}
%
%---------------------------------------------------------------------
%             Justificación
%---------------------------------------------------------------------
\section{Justificación}
\subsection{Conveniencia}

El presente trabajo se justifica por la necesidad de conocer el efecto del uso
de las herramientas de la computación en la nube en la gestión de las MiPYME
pertenecientes al Centro de Desarrollo Empresarial del Cusco y de esta forma
generar herramientas de apoyo a la gestión lo que permitira lograr su crecimiento.

%% Incluir parrafos de

\subsection{Relevancia social}
Las MiPYMES, como ya se menciono en las secciones anteriores de este capítulo,
tienen problemas en el acceso a las tecnologías de información por diversos motivos
lo que retrasa su crecimiento. Al realizar este estudio se implementaran e implantaran
herramientas de computación en la nube que permitiran mejorar los procesos de gestión
más importantes a un costo razonable que les permita desarrollarse y crecer.

\subsection{Implicancias pr'acticas}
Las herramientas que se implementaran en el presente trabajo mejorarán los procesos
de gestión de las MiPYMEs pertenecientes al Centro de Desarrollo Empresarial del Cusco.
Además los instrumentos elaborados en este trabajo serviran de base para estudios
similares donde se pretenda implementar tecnologias nuevas en entornos empresariales
similares.

%\subsection{Valor teorico}
%\subsection{Utilidad metodológica}

%---------------------------------------------------------------------
%             Objetivos
%---------------------------------------------------------------------
\section{Objetivos de Investigación}
\subsection{Objetivo General}
Determinar el efecto del uso de la computación en la nube en la gestión de
las micro, pequeñas y medianas empresas pertenecientes al Centro de desarrollo
empresarial del Cusco.
\subsection{Objetivos Especificos}
\begin{enumerate}[a.]
\item Determinar el nivel de gestión de las MiPYMES pertenecientes al Centro de
Desarrollo Empresarial del Cusco antes del uso de las herramientas de la
computación en la nube.
\item Determinar el nivel de gestión de las MiPYMES pertenecientes al Centro de
Desarrollo Empresarial del Cusco después del uso de las herramientas de la
computación en la nube.
\end{enumerate}
%---------------------------------------------------------------------
%             Delimitación
%---------------------------------------------------------------------
\section{Delimitación del estudio}
\subsection{Delimitación espacial}
% Colocar una descripcion de la CDE en esta parte
Este trabajo de investigación se realizara en las MiPYMES pertenencientes al
Centro de Desarrollo Empresarial, la cual agrupa a micro, pequeñas y
medianas empresas de distintos rubros de la ciudad del Cusco.

\subsection{Delimitación temporal}
Este trabajo se desarrollará en el periodo comprendido entre los meses de abril
y septiembre del año 2017.
