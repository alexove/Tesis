%---------------------------------------------------------------------
%             Cap\'itulo 1
%---------------------------------------------------------------------

\chapter{Introducci'on}

%---------------------------------------------------------------------
%            Planteamiento del Problema
%---------------------------------------------------------------------
\section{Planteamiento del problema}
Las micro, peque\~nas y medianas empresas, en adelante MiPYMEs, ejercen un papel
esencial en la economia peruana porque son una de las principales fuentes de empleo
del pa\'is, disminuyen la pobreza y dan un aporte importante al PBI.

Antes de iniciar con nuestros an\'alisis debemos de tener en cuenta que las MiPYMEs
seg\'un el \cite{produce2} est\'an divididas en 3 estratos de acuerdo
a la Ley Nro. 30056 que los clasifica de acuerdo al promedio de ventas anuales
(en UIT) en:
\begin{itemize}
  \item Micro empresa, hasta 150 UIT.
  \item Peque\~na empresa, de 150 a 1700 UIT.
  \item Mediana empresa, de 1700 a 2300 UIT.
\end{itemize}

Adem\'as debemos tener en cuenta que las MiPYMEs son la forma de empresa predominante
en el pa\'is que seg\'un el \cite{produce} la mayor\'ia de las empresas son
microempresas (95,0\%). El estrato de las PYME (peque\~nas y medianas empresas) presenta una baja participaci\'on,
con 4,3\% de peque\~nas empresas y 0,2\% de medianas empresas. Siendo las regiones
con mayor n\'umero de MiPYME son Lima, Arequipa, La Libertad, Cusco y Piura, las que
en conjunto constituyen el 66,3\% del total de las MiPYME peruanas.

Tambien es importante saber cuales son los rubros en los que se concentra la mayor cantidad de empresas,
al respecto \cite{produce} se\~nala que los rubros dominantes son el comercio (44,3\%) y servicios (41,1\%).

Claramente se denota la importancia de las MiPYME en la econom\'ia nacional debido a su aporte a la
producci\'on nacional que seg\'un \cite{arbulu} fue de 42\% el a\~no 2005 y por
su potencial de absorci\'on de empleo que fue cerca del 88\%.

Adem\'as \cite{arbulu} agrega que tal es la importancia de las MiPYMES que \'estas aportar\'on
49\% al PBI el a\~no 2005, mientras el que \citep{produce} indica que el aporte
de las MiPYMES al PBI fue de 40\% el a\~no 2013 lo que significa un descenso significativo,
sin embargo ambos casos porcentajes no dejan de ser importantes.

Respecto a las MiPYMES y el empleo \cite{produce2} muestra que 10 de cada 100 personas
de la Poblaci\'on Economicamente Activa (PEA) son conductoras de una MiPYME formal.
Adem\'as agrega que \'estas gener\'an el 60\% de la PEA ocupada. El 83,5\% de las
MiPYME formales tienen hasta cinco trabajadores, esta estructura no presenta cambios
sustanciales respecto a a\~nos anteriores y es explicado por el gran n\'umero de
MiPYME que se constituyen como personas naturales y presentan bajos niveles de
ventas. Sin embargo, esta proporci\'on var\'ia de acuerdo con el estrato empresarial:
el 86,7\% de las microempresas, 14,4\% de las peque\~nas empresas y 4,3\% de las
medianas empresas tienen como m\'aximo cinco trabajadores.

Gracias a los parrafos anteriores, se puede notar la importancia de las MiPYMEs en
la economia nacional tanto en la parte productiva, generaci\'on de divisas y empleo.
Sin embargo, existen problemas que no permiten el desarrollo de este tipo de empresas
y que el pa\'is no se beneficie adecuadamente del trabajo de \'estas. Uno de los problemas que aqueja a
las MiPYMEs es la informalidad. \cite{loayza} precisa que la informalidad est\'a
constituido por el conjunto de empresas, trabajadores y actividades que operan
fuera de los marcos legales y normativos que rigen la actividad econ\'omica. Por
lo tanto, pertenecer al sector informal supone estar al margen de las cargas
tributarias y normas legales, pero tambi\'en implica no contar con la protecci\'on
y los servicios que el Estado puede ofrecer.

Seg\'un \cite{arbulu} el desarrollo de la MiPYME y del sector informal urbano
en el Per\'u ha sido un fen\'omeno caracter\'istico de las \'ultimas dos
d\'ecadas, debido al acelerado proceso de migraci\'on y urbanizaci\'on que
sufrieron muchas ciudades, la aparici\'on del autoempleo y de una gran cantidad
de unidades econ\'omicas de peque\~na escala, frente a las limitadas fuentes
de empleo asalariada y formal para el conjunto de integrantes de la PEA (Poblaci\'on
Economicamente Activa).

\cite{loayza} se\~nala desde un punto de vista conceptual que la informalidad surge
cuando los costos de circunscribirse al marco legal y normativo de un pa\'is son
superiores a los beneficios que ello conlleva. La formalidad involucra costos
tanto en t\'erminos de ingresar a este sector - largos,complejos y costosos procesos
de inscripci\'on y registro - como en t\'erminos de permanecer dentro del mismo -pago
de impuestos, cumplir las normas referidas a beneficios laborales
y remuneraciones, manejo ambiental, salud, entre otros. A lo que \citep{penaranda}
agrega que existen una diversidad de enfoques que explican la presencia de la
informalidad en un sistema econ\'omico, como las barreras burocr\'aticas y los sobrecostos
laborales y tributarios, as\'i como tambi\'en los distintos mecanismos a trav\'es de
los cuales esta afecta la productividad y el potencial de crecimiento de una
econom\'ia. En el accionar de una econom\'ia informal se encuentran empresas que
buscan eludir el control del Estado, manteniendo un tama\~no inferior al \'optimo
para gozar de beneficios tributarios o laborales, empleando mecanismos irregulares
para la compra de bienes y servicios e incluso destinando recursos financieros
para encubrir actividades ilegales.

Respecto al n\'umero de MiPYMEs informales \cite{produce2} se\~nala que existe una
amplia heterogeneidad en las dimensiones de la informalidad, lo que dificulta
encontrar el n\'umero de empresas formales e informales de una manera \'unica y precisa.
Al respecto \citep{produce2} cita a (Diaz, 2014) que sintetiza en dos dimensiones
la informalidad: informalidad laboral e informalidad tributaria. En el caso de la
primera se distingue varios criterios (rasgos) para identificar las obligaciones
propias de una relaci\'on laboral como el acceso a un seguro de salud, una pensi\'on
de jubilaci\'on, gratificaciones y a un contrato de trabajo. En el caso de la segunda,
tambi\'an distingue criterios, como la tenencia de RUC de la empresa, si esta tiene
un sistema de contabilidad, y si se encuentra registrada como persona jur\'idica.

Pese a las dificultades que representa determinar el numero exacto de empresas
informales \cite{produce2} estim\'o que el a\~no 2010 existia un 30,4\% de empresas
formales frente a un 69,6\% de empresas informales (estimado), sin embargo las
cifras el a\~no 2014 mostraron que el porcentaje de empresas formales aumento a 43,9\%
y el porcentaje de empresas informales descendio a 56,1\% que a\'un sigue siendo
un numero bastante alto de informalidad.

Respecto a la informalidad \cite{inei2} indica que el mayor n\'umero de unidades
productivas informales se concentra en la actividad Agropecuaria (33,8\%), le
siguen las actividades de Comercio (23,9\%), Transportes (12,2\%) y Otros servicios (10,9\%).

\cite{perucamaras} expone un an\'alisis sobre la informalidad en la Macro Regi\'on
Sur de nuestro pa\'is donde el 79,1\% de la poblaci\'on ocupada labora en la informalidad,
mientras que solo el 20,9\% son trabajadores formales, agregando adem\'as que
2'013,914 personas en esta parte del pa\'is cuentan con empleos que no estan sujetos
a la legislaci\'on laboral nacional o que no pertenecen al sector formal de la
economia. Respecto a la actividad economica, se observa que en los sectores agropecuario
y pesquero la tasa de informalidad es de 99,2\%. Estas actividades concentran al
32,2\% de la Poblaci\'on Economicamente Activa (PEA) ocupada de esta macro regi\'on.

En la regi\'on del Cusco, seg\'un \cite{perucamaras}, el 83\% de la poblaci\'on
ocupada labora en la informalidad, mientras que solo el 17\% son trabajadores formales.

\cite{perucamaras} tambi\'en agrega que seg\'un la actividad econ\'omica, la mayor
fuerza laboral est\'a concentrada en los sectores agropecuario y pesquero (42\% de
la PEA ocupada), cuya tasa de informalidad es de 100\%. Los sectores de comercio
y servicios registran tasas de informalidad de 83,6\% y 78,2\%, respectivamente.
Dichas actividades representan el 15,5\% y 21,1\% de la PEA ocupada, respectivamente.
En esta regi\'on el 43,1\% de la poblaci\'on ocupada trabaja como independiente no
profesional y la informalidad alcanza al 91,8\%. Mientras que el 21,4\% se desempe\~na
como trabajador familiar no remunerado, actividad 100\% informal, y el 16\% labora en
microempresas, siendo el 83,1\% informal.

Respecto al uso de las Tecnolog\'ias de Informaci\'on y Comunicaciones (TIC) en las
MiPYMES del Per\'u, \cite{inei1} se\~nala que el 67,0\% de las Micro y Peque\~na Empresas
manifestaron contar por lo menos con una computadora de escritorio, el 48,8\% un
equipo multifuncional, el 35,1\% una computadora port\'atil, el 27,3\% una impresora,
el 23,5\% un tel\'efono con acceso a internet (Smartphone), el 8,6\% un esc\'aner y
el 6,0\% una fotocopiadora. A  nivel  de  ciudad,  se  observa  que  en la ciudad
del Cusco el 50\% de MiPYMEs cuentan con una computadora de escritorio, 31,7\% con
una computadora portatil, 36,6\% con un equipo multifuncional, 18,3\% con una impresora,
6,1\% con un sc\'aner, 1,2\% con una fotocopiadora, 12,2\% con un smart phone con acceso
Internet.

Respecto al acceso a los servicios inform\'aticos que disponen las computadoras de
las MiPYMEs \cite{inei1} indica que El 90,9\% declararon contar con servicios de
internet, el 0,9\% declar\'o tener intranet y el 9,0\% declar\'o no contar con
servicios inform\'aticos. Las ciudades de Cusco con 37,3\% y Juliaca con 36,6\%,
presentan los m\'as altos porcentajes de empresas que no cuentan con servicios
inform\'aticos en sus computadoras.

Otro aspecto importante de las MiPYMEs y su relaci\'on con las TIC es el uso de los
sistemas de de gesti\'on, al respecto \citep{inei1} se\~nala que el 11,7\% contaban
con sistemas contable - tributario y el 7,1\% de ventas. En orden de importancia
le siguen los sistemas de producci\'on con 5,0\%, personal con 2,7\% y soporte inform\'atico
con 2,3\%. Es oportuno resaltar que el 52,6\% de Micro y Peque\~na Empresa no contaban
con ning\'un tipo de sistema de gesti\'on. En la ciudad del Cusco los resultados
muestran que el 6,1\% de MiPYMEs cuentan con un sistema de informaci\'on contable-tributario,
2,4\% con sistema de ventas, 2,4\% sistema de personal, 1,2\% sistema de finanzas,
1,2\% sistema log\'istico, 2,4\% sistemas de producci\'on, 2,4\% sistemas de soporte
inform\'atico y un preocupante 52,4\% no tiene ningun sistema de informaci\'on ni de
gesti\'on.

A los datos proporcionados anteriormente respecto al uso de las TIC en las MiPYMEs
\cite{ipsos} proporciona datos adicionales como la gesti\'on de las TIC dentro
de la empresa y al uso de la computaci\'on en la nube en la gesti\'on del negocio.

Con respecto a la gesti\'on de las TIC en las mypes \citep{ipsos} muestra que un
45\% de las empresas delega el manejo de las TIC a los empleados que los utilizan
(autoservicio), el 14\% lo administra un proveedor externo de TI, un 13\% son administrados
por un departamento interno de TI y un 28\% no maneja recursos de TI. Respecto a
esto podemos decir que el segundo y tercer caso son una forma adecuada de manejar
los recursos de TI, sin embargo estan en un segundo plano.

En adici\'on \cite{ipsos} muestra que en un 71\% de MiPYMEs el due\~no, fundador o director
ejecutivo es el responsable de decidir la compra y uso de TI, el 7\% lo decide
gerente especializado de TI, director de tecnolog\'ia (CTO) o un director de informaci\'on
(CIO), en un 3\% de casos lo decide un comit\'e de tecnolog\'ia, en un 2\% lo decide
cada empleado, en un 16\% lo deciden otros y 1\% no sabe. Respecto a esto debemos
decir que el segundo y tercer caso son los m\'as adecuados y el primero puede ser la
causa de la mayoria de problemas de las MiPYMEs porque el criterio de ahorro de dinero
se anteponen a la conveniencia de dicha tecnolog\'ia.

Con respecto al nivel de inversi\'on en TIC \cite{ipsos} se\~nala que un 44\% de MiPYMEs
gasta muy poco en las TIC, seguido de un 34\% gasta solo en lo necesario en las TIC,
un 10\% gasta un poco menos de lo necesario, otro 10\% gasta algo m\'as de lo necesario y
1\% gasta demasiado en TIC. Obviamente lo preocupante en este resultado es que la
mayoria de empresas no considera importante a las TIC para la gesti\'on de su empresa.

Por otro lado \cite{ipsos} se\~nala que ante la frase "Entiendo completamente el
t\'ermino computaci\'on en la nube para los negocios", un 28\% esta muy de acuerdo, un 32\%
esta en algo de acuerdo, 12\% es neutral (ni de acuerdo ni en desacuerdo), 12\%
algo en desacuerdo, 11\% muy en desacuerdo y 5\% no sabe. Esto indica que un grupo
considerable de entrevistados conoce la computaci\'on en la nube lo que aporta a que
dentro de un tiempo puedan adoptarla.

Ante la propuesta de que las empresas destinen mayor parte de su presupuesto a
soluciones de computaci\'on en la nube \cite{ipsos} indica que un 18\% de MiPYMEs
considera que sera muy probable, 46\% probable, 20\% no muy probable, 13\% nada
probable y 3\% no sabe. Esto indica que estas empresas requeriran m\'as soluciones
de este tipo en el futuro.

En resumen, vemos en los parrafos precedentes que se da de manifiesto la importancia
de las MiPYMEs, debido a que generan divisas, aportan al PBI del pa\'is y generan
un porcentaje importante del empleo a nivel nacional. Sin embargo, tambien se da
de manifiesto que uno de los principales problemas de estas empresas es la informalidad
que impacta en la recaudaci\'on de impuestos y causa la informalidad en el empleo
lo que conlleva a que los colaboradores (empleados) no gocen de los beneficios de la formalidad.
Por otro lado, a pesar de que la mayoria de MiPYMEs cuenta con distintos equipos
inform\'aticos y acceso a Internet, esto no constituye una ayuda o ventaja competitiva
lo que queda de manifiesto en la ausencia del uso de sistemas de informaci\'on en
apoyo a la gesti\'on.

Todos estos hechos traen como consecuencia que las MiPYMEs se ver\'an limitadas
significativamente en su crecimiento ya que no aprovechan las TIC adecuadamente
que les ayudaria a simplificar las labores de gesti\'on (labores financieras, contables,
recursos humanos, etc) y tambien tener acceso a los mercados nacionales e internacionales.

Es necesario que las empresas tengan acceso a las TIC para que puedan mejorar la
gesti\'on sus recursos y planteen tacticas de crecimiento reales. La computaci\'on
en la nube ofrece la posibilidad de tener acceso obicuo (en todo lugar) a bajo
costo a las TIC, en especial a los sistemas de gesti\'on para la empresa.

%---------------------------------------------------------------------
%            Formulaci\'on del problema
%---------------------------------------------------------------------
\section{Formulaci'on del problema}
\subsection{Problema General}
?`Cual es el efecto del uso de las herramientas de la computaci\'on en la nube
en la gesti\'on de las MiPYME pertenecientes al Centro de Desarrollo Empresarial
del Cusco?
\subsection{Problemas Espec'ificos}
\begin{enumerate}[a.]
\item ?`Cu\'al es el nivel de gesti\'on de las MiPYMES pertenecientes al Centro
de Desarrollo Empresarial del Cusco antes del uso de herramientas de la computaci\'on
en la nube?
\item ?`Cu\'al es el nivel de gesti\'on de las MiPYMES pertenecientes al Centro
de Desarrollo Empresarial del Cusco despu\'es del uso de herramientas de la computaci\'on
en la nube?
\end{enumerate}

%\subsubsection{Alcance Descriptivo}
%\subsubsection{Alcance Correlacional}
%\subsubsection{Alcance Explicativo}
%
%---------------------------------------------------------------------
%             Justificaci\'on
%---------------------------------------------------------------------
\section{Justificaci'on}
\subsection{Conveniencia}

El presente trabajo se justifica por la necesidad de conocer el efecto del uso
de las herramientas de la computaci\'on en la nube en la gesti\'on de las MiPYME
pertenecientes al Centro de Desarrollo Empresarial del Cusco y de esta forma
generar herramientas de apoyo a la gesti\'on lo que permitira lograr su crecimiento.

%% Incluir parrafos de

\subsection{Relevancia social}
Las MiPYMES, como ya se menciono en las secciones anteriores de este cap\'itulo,
tienen problemas en el acceso a las tecnolog\'ias de informaci\'on por diversos motivos
lo que retrasa su crecimiento. Al realizar este estudio se implementaran e implantaran
herramientas de computaci\'on en la nube que permitiran mejorar los procesos de gesti\'on
m\'as importantes a un costo razonable que les permita desarrollarse y crecer.

\subsection{Implicancias pr'acticas}
Las herramientas que se implementaran en el presente trabajo mejorar\'an los procesos
de gesti\'on de las MiPYMEs pertenecientes al Centro de Desarrollo Empresarial del Cusco.
Adem\'as los instrumentos elaborados en este trabajo serviran de base para estudios
similares donde se pretenda implementar tecnologias nuevas en entornos empresariales
similares.

%\subsection{Valor teorico}
%\subsection{Utilidad metodol'ogica}

%---------------------------------------------------------------------
%             Objetivos
%---------------------------------------------------------------------
\section{Objetivos de Investigaci'on}
\subsection{Objetivo General}
Determinar el efecto del uso de la computaci\'on en la nube en la gesti\'on de
las micro, peque\~nas y medianas empresas pertenecientes al Centro de desarrollo
empresarial del Cusco.
\subsection{Objetivos Especificos}
\begin{enumerate}[a.]
\item Determinar el nivel de gesti\'on de las MiPYMES pertenecientes al Centro de
Desarrollo Empresarial del Cusco antes del uso de las herramientas de la
computaci\'on en la nube.
\item Determinar el nivel de gesti\'on de las MiPYMES pertenecientes al Centro de
Desarrollo Empresarial del Cusco despu\'es del uso de las herramientas de la
computaci\'on en la nube.
\end{enumerate}
%---------------------------------------------------------------------
%             Delimitaci\'on
%---------------------------------------------------------------------
\section{Delimitaci'on del estudio}
\subsection{Delimitaci'on espacial}
% Colocar una descripcion de la CDE en esta parte
Este trabajo de investigaci\'on se realizara en las MiPYMES pertenencientes al
Centro de Desarrollo Empresarial, la cual agrupa a micro, peque\~nas y
medianas empresas de distintos rubros de la ciudad del Cusco.

\subsection{Delimitaci'on temporal}
Este trabajo se desarrollar\'a en el periodo comprendido entre los meses de abril
y septiembre del a\~no 2017.
