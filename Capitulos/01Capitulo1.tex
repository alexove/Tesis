%---------------------------------------------------------------------
%
%                          Capítulo 1
%
%---------------------------------------------------------------------

\chapter{Introducci'on}

\section{Planteamiento del problema}
Durante muchos a\~nos se ha dicho por distintos medios de comunicaci\'on que
las MiPYME (Micro, peque\~nas y medianas empresas) son la expresi\'on m\'as pura
del emprendimiento del pa\'is, ya que constituyen la mayor cantidad de empresas
del pais y las que mayor aporte hacen
Seg'un \citep{produce} las microempresas constituyen el 95\% de empresas del
pa'is, mientras que solo el 4\% son peque'nas y medianas.

%Citar fuentes que acrediten los porcentajes de aporte al PBI y la
%empleabilidad%
Las PYMES en los \'ultimos a\~nos se han convertido en una de las mayores
preocupaciones del estado debido al aporte que hacen al PBI y la
generaci\'on empleo a nivel nacional.
%Citar la fuente para acreditar su origen%

El origen de las PYME's se debe en mayor medida a los procesos de
migraci\'on y urbanizaci\'on de las capitales de distrito, provincia,
regi\'on y pais que se han estado dando en los ultimos a\~nos. Los
migrantes, de zonas rurales en su mayoria, al tener limitado acceso
a las fuentes de empleo asalariada y formal se vieron obligadas a
realizar actividades economicas de peque\~na escala propiciando un
fenomeno denominado autoempleo que ...
\section{Formulaci'on del problema}
\subsection{Problema General}
?`Cual es el efecto del uso de las herramientas de la computaci\'on en la nube
en la gesti'on de las MiPYME pertenecientes al Centro de Desarrollo Empresarial
del Cusco?
\subsection{Problemas Espec'ificos}
\begin{enumerate}[a.]
\item ?`Como es la gesti\'on de las MiPYMES pertenecientes al Centro de
Desarrollo Empresarial del Cusco?
\item ?`Cual es el nivel de adopci\'on de TIC en general en las MiPYMES
pertenecientes al Centro de Desarrollo Empresarial del Cusco?
\item ?`Cual es el nivel de adopci\'on de herramientas de la computaci\'on en la
nube en las MiPYMES pertenecientes al Centro de Desarrollo Empresarial del Cusco?
\item ?`Como es la influencia del uso de la computaci\'on en la nube en la
gesti\'on de las MiPYMES pertenecientes al Centro de Desarrollo Empresarial
del Cusco?
\end{enumerate}
%
%\subsubsection{Alcance Descriptivo}
%\subsubsection{Alcance Correlacional}
%\subsubsection{Alcance Explicativo}
%
\section{Justificaci'on}
\subsection{Conveniencia}

\subsection{Relevancia social}
%Las MiPYMES, como ya se menciono en las secciones anteriores de este cap\'itulo,
%tienen problemas en el acceso a las tecnolog\'ias de informaci\'on por diversos motivos
%que retrasa su crecimiento. Al realizar este estudio

%\subsection{Implicancias pr'acticas}
%\subsection{Valor teorico}
%\subsection{Utilidad metodol'ogica}
\section{Objetivos de Investigaci'on}
\subsection{Objetivo General}
Determinar el efecto del uso de la computaci\'on en la nube en la gesti\'on de
las micro, peque\~nas y medianas empresas pertenecientes al Centro de desarrollo
empresarial del Cusco.
\subsection{Objetivos Especificos}
\begin{enumerate}[a.]
\item Determinar como es la gesti\'on de las MiPYMES asociadas al Centro de
Desarrollo Empresarial del Cusco.
\item Determinar el nivel de adopci\'on de la computaci\'on en la nube en las
MiPYMES asociadas al Centro de Desarrollo Empresarial del Cusco.
\item Determinar el nivel de adopci\'on de TIC en general en las MiPYMES
asociadas al Centro de Desarrollo Empresarial del Cusco.
\item Estimar el grado de influencia del uso de la computaci\'on en la nube
en la gesti\'on de las MiPYMES asociadas al Centro de Desarrollo Empresarial
del Cusco
\end{enumerate}
\section{Delimitaci'on del estudio}
\subsection{Delimitaci'on espacial}
Este trabajo se desarrolla en
\subsection{Delimitaci'on temporal}
Este trabajo se desarrollara en
%-------------------------------------------------------------------
%\section*{\NotasBibliograficas}
%-------------------------------------------------------------------
%\citep{ldesc2e} esto es una cita al inicio.

%Citamos algo para que aparezca en la bibliograf\'ia\ldots
%\citep{ldesc2e}

%\medskip

%Y también ponemos el acr�nimo \ac{CVS} para que no cruja.

%Ten en cuenta que si no quieres acr�nimos (o no quieres que te falle la
%compilaci�n en ``release'' mientras no tengas ninguno) basta con que no definas
%la constante \verb+\acronimosEnRelease+ (en \texttt{config.tex}).

% Variable local para emacs, para  que encuentre el fichero maestro de
% compilaci�n y funcionen mejor algunas teclas r�pidas de AucTeX
%%%
%%% Local Variables:
%%% mode: latex
%%% TeX-master: "../Tesis.tex"
%%% End:
