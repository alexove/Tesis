%---------------------------------------------------------------------
%                          Cap\'itulo 1
%---------------------------------------------------------------------

\chapter{Introducci'on}

%---------------------------------------------------------------------
%                        Planteamiento del Problema
%---------------------------------------------------------------------
\section{Planteamiento del problema}
Seg\'un \cite{arbulu} el desarrollo de la PYME (Peque\~na y  Micro  Empresa) y
del  sector  informal  urbano  en  el  Per\'u  ha  sido un fen\'omeno caracter\'istico
de las \'ultimas  dos  d\'ecadas,  debido  al acelerado proceso de migraci\'on y urbanizaci\'on
que  sufrieron  muchas ciudades, la aparici\'on del autoempleo y de una gran cantidad
de  unidades  econ\'omicas  de  peque\~na  escala,  frente  a  las  limitadas  fuentes
de  empleo  asalariada y formal para el conjunto de integrantes de la PEA.

Adem'as el \cite{produce} indica que la estructura empresarial peruana del 2015 no
presenta cambios sustanciales respecto de lo que ha venido ocurriendo en el pasado:
la gran mayor\'ia de las empresas son microempresas (95,0\%). El estrato de las
PYME presenta una baja participaci\'on, con 4,3\% de peque\~nas empresas y 0,2\% de
medianas empresas. Claramente se denota la importancia de las MiPYME en la econom\'ia
nacional debido a su aporte a la producci\'on nacional que seg\'un \citep{arbulu} fue de 42\%
el a\~no 2005 y por su potencial de absorci\'on de empleo que fue cerca del 88\%.

Sin embargo, las ultimas cifras indican que la mayoria de MiPYMES son informales
%Citar fuentes que acrediten los porcentajes de aporte al PBI y la
%empleabilidad%

%---------------------------------------------------------------------
%                       Formulaci\'on del problema
%---------------------------------------------------------------------
\section{Formulaci'on del problema}
\subsection{Problema General}
?`Cual es el efecto del uso de las herramientas de la computaci\'on en la nube
en la gesti'on de las MiPYME pertenecientes al Centro de Desarrollo Empresarial
del Cusco?
\subsection{Problemas Espec'ificos}
\begin{enumerate}[a.]
\item ?`Cu\'al es el nivel de gesti\'on de las MiPYMES pertenecientes al Centro
de Desarrollo Empresarial del Cusco antes del uso de herramientas de la computaci\'on
en la nube?
\item ?`Cu\'al es el nivel de gesti\'on de las MiPYMES pertenecientes al Centro
de Desarrollo Empresarial del Cusco despu\'es del uso de herramientas de la computaci\'on
en la nube?
\end{enumerate}
%
%\subsubsection{Alcance Descriptivo}
%\subsubsection{Alcance Correlacional}
%\subsubsection{Alcance Explicativo}
%
%---------------------------------------------------------------------
%                          Justificaci\'on
%---------------------------------------------------------------------
\section{Justificaci'on}
\subsection{Conveniencia}

\subsection{Relevancia social}
%Las MiPYMES, como ya se menciono en las secciones anteriores de este cap\'itulo,
%tienen problemas en el acceso a las tecnolog\'ias de informaci\'on por diversos motivos
%que retrasa su crecimiento. Al realizar este estudio

%\subsection{Implicancias pr'acticas}
%\subsection{Valor teorico}
%\subsection{Utilidad metodol'ogica}

%---------------------------------------------------------------------
%                          Objetivos
%---------------------------------------------------------------------
\section{Objetivos de Investigaci'on}
\subsection{Objetivo General}
Determinar el efecto del uso de la computaci\'on en la nube en la gesti\'on de
las micro, peque\~nas y medianas empresas pertenecientes al Centro de desarrollo
empresarial del Cusco.
\subsection{Objetivos Especificos}
\begin{enumerate}[a.]
\item Determinar el nivel de gesti\'on de las MiPYMES pertenecientes al Centro de
Desarrollo Empresarial del Cusco antes del uso de las herramientas de la
computaci\'on en la nube.
\item Determinar el nivel de gesti\'on de las MiPYMES pertenecientes al Centro de
Desarrollo Empresarial del Cusco despu\'es del uso de las herramientas de la
computaci\'on en la nube.
\end{enumerate}
%---------------------------------------------------------------------
%                          Delimitaci\'on
%---------------------------------------------------------------------
\section{Delimitaci'on del estudio}
\subsection{Delimitaci'on espacial}
% Colocar una descripcion de la CDE en esta parte
Este trabajo de investigaci\'on se realizo en las MiPYMES pertenencientes al
Centro de Desarrollo Empresarial, la cual agrupa a micro, peque\~nas y
medianas empresas de distintos rubros de la ciudad del Cusco.

\subsection{Delimitaci'on temporal}
Este trabajo se desarrollar\'a en el periodo comprendido entre el .....
