%---------------------------------------------------------------------
%
%                          Capítulo 1
%
%---------------------------------------------------------------------

\chapter{Introducci'on}

\section{Planteamiento del problema}
La existencia de la MiPYME se debe principalmente a la necesidad de las personas
de encontrar un empleo, o en su defecto un empleo bien remunerado acorde a sus
capacidades o expectativas.

Durante muchos a\~nos se ha dicho por distintos medios de comunicaci\'on que
las MiPYME (Micro, peque\~nas y medianas empresas) son la expresi\'on m\'as pura
del emprendimiento del pa\'is, ya que constituyen la mayor cantidad de empresas
del pais y las que mayor aporte hacen
Seg'un \citep{produce} las microempresas constituyen el 95\% de empresas del
pa'is, mientras que solo el 4\% son peque'nas y medianas.

%Citar fuentes que acrediten los porcentajes de aporte al PBI y la
%empleabilidad%
Las PYMES en los \'ultimos a\~nos se han convertido en una de las mayores
preocupaciones del estado debido al aporte que hacen al PBI y la
generaci\'on empleo a nivel nacional\citep{arbulu}.
%Citar la fuente para acreditar su origen%

El origen de las PYME's se debe en mayor medida a los procesos de
migraci\'on y urbanizaci\'on de las capitales de distrito, provincia,
regi\'on y pais que se han estado dando en los ultimos a\~nos. Los
migrantes, de zonas rurales en su mayoria, al tener limitado acceso
a las fuentes de empleo asalariada y formal se vieron obligadas a
realizar actividades economicas de peque\~na escala propiciando un
fenomeno denominado autoempleo que ...
\section{Formulaci'on del problema}
\subsection{Problema General}
?`Cual es el efecto del uso de las herramientas de la computaci\'on en la nube
en la gesti'on de las MiPYME pertenecientes al Centro de Desarrollo Empresarial
del Cusco?
\subsection{Problemas Espec'ificos}
\begin{enumerate}[a.]
\item ?`Cu\'al es el nivel de gesti\'on de las MiPYMES pertenecientes al Centro
de Desarrollo Empresarial del Cusco antes del uso de herramientas de la computaci\'on
en la nube?
\item ?`Cu\'al es el nivel de gesti\'on de las MiPYMES pertenecientes al Centro
de Desarrollo Empresarial del Cusco despu\'es del uso de herramientas de la computaci\'on
en la nube?
\end{enumerate}
%
%\subsubsection{Alcance Descriptivo}
%\subsubsection{Alcance Correlacional}
%\subsubsection{Alcance Explicativo}
%
\section{Justificaci'on}
\subsection{Conveniencia}

\subsection{Relevancia social}
%Las MiPYMES, como ya se menciono en las secciones anteriores de este cap\'itulo,
%tienen problemas en el acceso a las tecnolog\'ias de informaci\'on por diversos motivos
%que retrasa su crecimiento. Al realizar este estudio

%\subsection{Implicancias pr'acticas}
%\subsection{Valor teorico}
%\subsection{Utilidad metodol'ogica}
\section{Objetivos de Investigaci'on}
\subsection{Objetivo General}
Determinar el efecto del uso de la computaci\'on en la nube en la gesti\'on de
las micro, peque\~nas y medianas empresas pertenecientes al Centro de desarrollo
empresarial del Cusco.
\subsection{Objetivos Especificos}
\begin{enumerate}[a.]
\item Determinar el nivel de gesti\'on de las MiPYMES pertenecientes al Centro de
Desarrollo Empresarial del Cusco antes del uso de las herramientas de la
computaci\'on en la nube.
\item Determinar el nivel de gesti\'on de las MiPYMES pertenecientes al Centro de
Desarrollo Empresarial del Cusco despu\'es del uso de las herramientas de la
computaci\'on en la nube.
\end{enumerate}
\section{Delimitaci'on del estudio}
\subsection{Delimitaci'on espacial}
% Colocar una descripcion de la CDE en esta parte
Este trabajo de investigaci\'on se realizo en las MiPYMES pertenencientes al
Centro de Desarrollo Empresarial, la cual agrupa a micro, peque\~nas y
medianas empresas de distintos rubros de la ciudad del Cusco.

\subsection{Delimitaci'on temporal}
