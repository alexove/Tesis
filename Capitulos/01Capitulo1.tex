%---------------------------------------------------------------------
%                          Cap\'itulo 1
%---------------------------------------------------------------------

\chapter{Introducci'on}

%---------------------------------------------------------------------
%                        Planteamiento del Problema
%---------------------------------------------------------------------
\section{Planteamiento del problema}
Las micro, peque\~nas y medianas empresas, en adelante MiPymes, ejercen un papel
esencial en la economia peruana porque son una de las principales fuentes de empleo
del pa\'is, disminuyen la pobreza y aportan al PBI.

Seg\'un el \cite{produce2} las empresas est\'an divididas en 3 estratos de acuerdo
a la Ley Nro. 30056 que los clasifica de acuerdo al promedio de ventas anuales
(en UIT) en:
\begin{itemize}
    \item Micro empresa, hasta 150 UIT.
    \item Peque\~na empresa, de 150 a 1700 UIT.
    \item Mediana empresa, de 1700 a 2300 UIT.
\end{itemize}

Adem\'as el \cite{produce} indica que la gran mayor\'ia de las empresas son
microempresas (95,0\%). El estrato de las PYME presenta una baja participaci\'on,
con 4,3\% de peque\~nas empresas y 0,2\% de medianas empresas. Siendo las regiones
con mayor n\'umero de MiPYME son Lima, Arequipa, La Libertad, Cusco y Piura, las que
en conjunto constituyen el 66,3\% del total de las MiPYME peruanas.

Tambien es importante conocer los rubros en los que concentra la mayor cantidad
de empresas, al respecto \citep{produce} los menciona en la tabla \ref{tabla_mypes}.

\begin{table}[htbp]
    \caption{MiPYMES formales, seg\'un sector econ\'omico, 2010 y 2015}
    \label{tabla_mypes}
    \centering
        \begin{tabular}{|p{2.1cm}|p{3cm}|p{3cm}|p{3cm}|}
            \hline
            \thead{Sector\\ econ\'omico} & \thead{MiPYME 2010} & \thead{MiPYME 2015} & \thead{VAP \\ 2010-2015} \\ \hline
            % Primera fila
            Comercio &
            547,651 ; 45.8 &
            745,295 ; 44.3 &
            6.4 \\
            \hline
            % Segunda fila
            Servicios &
            462,850 ; 38.7 &
            692,221 ; 41.1 &
            8.4 \\
            \hline
            % Tercera fila
            Manufactura &
            121,242 ; 10.1 &
            148,732 ; 8.8 &
            4.2 \\
            \hline
            % Cuarta fila
            Construcci\'on &
            31,898 ; 2.7 &
            55,083 ; 3.3 &
            11.5 \\
            \hline
            % Quinta fila
            Agropecuario &
            22,202 ; 1.9 &
            24,184 ; 1.4 &
            1.7 \\
            \hline
            % Sexta fila
            Miner\'ia &
            6,375  ; 0.5 &
            13,669 ; 0.8 &
            16.5 \\
            \hline
            % Septima fila
            Pesca &
            3,493 ; 0.3 &
            3,497 ; 0.2 &
            0.0 \\
            \hline
            % Octava fila
            Total &
            1,195,711 ; 100 &
            1,682,681 ; 100 &
            7.1 \\
            \hline
        \end{tabular}
        \thead{Fuente: Sunat, Registro \'Unico del Contribuyente 2010 y 2015 \\
        Elaboraci\'on: PRODUCE - Direcci\'on de Estudios Econ\'omicos de Mype e Industria (DEMI)I}
\end{table}

Seg\'un \cite{arbulu} el desarrollo de la MiPYME y del  sector  informal  urbano
en  el  Per\'u  ha  sido un fen\'omeno caracter\'istico de las \'ultimas  dos
d\'ecadas,  debido  al acelerado proceso de migraci\'on y urbanizaci\'on que
sufrieron  muchas ciudades, la aparici\'on del autoempleo y de una gran cantidad
de  unidades  econ\'omicas  de  peque\~na  escala,  frente  a  las  limitadas  fuentes
de  empleo  asalariada y formal para el conjunto de integrantes de la PEA (Poblaci\'on
Economicamente Activa).


 Claramente se denota
la importancia de las MiPYME en la econom\'ia nacional debido a su aporte a la
producci\'on nacional que seg\'un \citep{arbulu} fue de 42\% el a\~no 2005 y por
su potencial de absorci\'on de empleo que fue cerca del 88\%.

\cite{arbulu} agrega que tal es la importancia de las MiPYMES que \'estas aportar\'on
49\% al PBI el a\~no 2005, mientras el que \citep{produce} indica que el aporte
de las MiPYMES al PBI fue de 40\% el a\~no 2013 lo que significa un descenso significativo,
sin embargo ambos casos porcentajes no dejan de ser importantes.

Respecto a las MiPYMES y el empleo \cite{produce2} muestra que 10 de cada 100 personas
de la Poblaci\'on Economicamente Activa (PEA) son conductoras de una MiPYME formal.
Adem\'as agrega que \'estas gener\'an el 60\% de la PEA ocupada.

Es indudable que las Mipymes se han convertido en los actores primordiales en la
econom\'ia peruana, pero es necesario indicar que a\'un no se encuentran listas
para participar en el comercio internacional pues sus niveles tecnol\'ogicos no
les permiten adaptarse a los nuevos flujos de informaci\'on.

Uno de los principales problemas de las Mypime peruanas es la informalidad.

%---------------------------------------------------------------------
%                       Formulaci\'on del problema
%---------------------------------------------------------------------
\section{Formulaci'on del problema}
\subsection{Problema General}
?`Cual es el efecto del uso de las herramientas de la computaci\'on en la nube
en la gesti'on de las MiPYME pertenecientes al Centro de Desarrollo Empresarial
del Cusco?
\subsection{Problemas Espec'ificos}
\begin{enumerate}[a.]
\item ?`Cu\'al es el nivel de gesti\'on de las MiPYMES pertenecientes al Centro
de Desarrollo Empresarial del Cusco antes del uso de herramientas de la computaci\'on
en la nube?
\item ?`Cu\'al es el nivel de gesti\'on de las MiPYMES pertenecientes al Centro
de Desarrollo Empresarial del Cusco despu\'es del uso de herramientas de la computaci\'on
en la nube?
\end{enumerate}
%
%\subsubsection{Alcance Descriptivo}
%\subsubsection{Alcance Correlacional}
%\subsubsection{Alcance Explicativo}
%
%---------------------------------------------------------------------
%                          Justificaci\'on
%---------------------------------------------------------------------
\section{Justificaci'on}
\subsection{Conveniencia}

\subsection{Relevancia social}
%Las MiPYMES, como ya se menciono en las secciones anteriores de este cap\'itulo,
%tienen problemas en el acceso a las tecnolog\'ias de informaci\'on por diversos motivos
%que retrasa su crecimiento. Al realizar este estudio

%\subsection{Implicancias pr'acticas}
%\subsection{Valor teorico}
%\subsection{Utilidad metodol'ogica}

%---------------------------------------------------------------------
%                          Objetivos
%---------------------------------------------------------------------
\section{Objetivos de Investigaci'on}
\subsection{Objetivo General}
Determinar el efecto del uso de la computaci\'on en la nube en la gesti\'on de
las micro, peque\~nas y medianas empresas pertenecientes al Centro de desarrollo
empresarial del Cusco.
\subsection{Objetivos Especificos}
\begin{enumerate}[a.]
\item Determinar el nivel de gesti\'on de las MiPYMES pertenecientes al Centro de
Desarrollo Empresarial del Cusco antes del uso de las herramientas de la
computaci\'on en la nube.
\item Determinar el nivel de gesti\'on de las MiPYMES pertenecientes al Centro de
Desarrollo Empresarial del Cusco despu\'es del uso de las herramientas de la
computaci\'on en la nube.
\end{enumerate}
%---------------------------------------------------------------------
%                          Delimitaci\'on
%---------------------------------------------------------------------
\section{Delimitaci'on del estudio}
\subsection{Delimitaci'on espacial}
% Colocar una descripcion de la CDE en esta parte
Este trabajo de investigaci\'on se realizo en las MiPYMES pertenencientes al
Centro de Desarrollo Empresarial, la cual agrupa a micro, peque\~nas y
medianas empresas de distintos rubros de la ciudad del Cusco.

\subsection{Delimitaci'on temporal}
Este trabajo se desarrollar\'a en el periodo comprendido entre el .....
