%---------------------------------------------------------------------
%                          Cap\'itulo 1
%---------------------------------------------------------------------

\chapter{Introducci'on}

%---------------------------------------------------------------------
%                        Planteamiento del Problema
%---------------------------------------------------------------------
\section{Planteamiento del problema}
Las micro, peque\~nas y medianas empresas, en adelante MiPymes, ejercen un papel
esencial en la economia peruana porque son una de las principales fuentes de empleo
del pa\'is, disminuyen la pobreza y aportan al PBI.

Seg\'un el \cite{produce2} las empresas est\'an divididas en 3 estratos de acuerdo
a la Ley Nro. 30056 que los clasifica de acuerdo al promedio de ventas anuales
(en UIT) en:
\begin{itemize}
    \item Micro empresa, hasta 150 UIT.
    \item Peque\~na empresa, de 150 a 1700 UIT.
    \item Mediana empresa, de 1700 a 2300 UIT.
\end{itemize}

Adem\'as el \cite{produce} indica que la mayor\'ia de las empresas son
microempresas (95,0\%). El estrato de las PYME presenta una baja participaci\'on,
con 4,3\% de peque\~nas empresas y 0,2\% de medianas empresas. Siendo las regiones
con mayor n\'umero de MiPYME son Lima, Arequipa, La Libertad, Cusco y Piura, las que
en conjunto constituyen el 66,3\% del total de las MiPYME peruanas.

\cite{produce2} Con respecto a la distribuci\'on de las Mipyme formales peruanas seg\'un el tipo de
contribuyente y el estrato empresarial, la gran mayor\'ia de las mismas se constituye
como persona natural (70,6\%). Sin embargo, este porcentaje var\'ia seg\'un el estrato
empresarial, donde el 73,2\% de las microempresas se constituyeron como persona
natural. Por el contrario, solo el 26,8\% de las microempresas se constituyen
como persona jur\'idica, en el caso de la peque\~na y mediana empresa el 85,7\% y
95,7\% se constituye como persona jur\'idica, respectivamente. Vale la pena mencionar
que en los tres estratos empresariales, la Sociedad An\'onima Cerrada es el tipo
de personer\'ia jur\'idica que m\'as destaca, seguida de la Empresas Individual de
Responsabilidad limitada.

Tambien es importante conocer los rubros en los que concentra la mayor cantidad
de empresas, al respecto \citep{produce} los menciona en la tabla \ref{tabla_mypes}.

\begin{table}[htbp]
    \caption{MiPYMES formales, seg\'un sector econ\'omico, 2010 y 2015}
    \label{tabla_mypes}
    \centering
        \begin{tabular}{|p{2.1cm}|p{3cm}|p{3cm}|p{3cm}|}
            \hline
            \thead{Sector\\ econ\'omico} & \thead{MiPYME 2010} & \thead{MiPYME 2015} & \thead{VAP \\ 2010-2015} \\ \hline
            % Primera fila
            Comercio &
            547,651 ; 45.8 &
            745,295 ; 44.3 &
            6.4 \\
            \hline
            % Segunda fila
            Servicios &
            462,850 ; 38.7 &
            692,221 ; 41.1 &
            8.4 \\
            \hline
            % Tercera fila
            Manufactura &
            121,242 ; 10.1 &
            148,732 ; 8.8 &
            4.2 \\
            \hline
            % Cuarta fila
            Construcci\'on &
            31,898 ; 2.7 &
            55,083 ; 3.3 &
            11.5 \\
            \hline
            % Quinta fila
            Agropecuario &
            22,202 ; 1.9 &
            24,184 ; 1.4 &
            1.7 \\
            \hline
            % Sexta fila
            Miner\'ia &
            6,375  ; 0.5 &
            13,669 ; 0.8 &
            16.5 \\
            \hline
            % Septima fila
            Pesca &
            3,493 ; 0.3 &
            3,497 ; 0.2 &
            0.0 \\
            \hline
            % Octava fila
            Total &
            1,195,711 ; 100 &
            1,682,681 ; 100 &
            7.1 \\
            \hline
        \end{tabular}
        \thead{Fuente: Sunat, Registro \'Unico del Contribuyente 2010 y 2015 \\
        Elaboraci\'on: PRODUCE - Direcci\'on de Estudios Econ\'omicos de Mype e Industria (DEMI)I}
\end{table}

Seg\'un \cite{arbulu} el desarrollo de la MiPYME y del  sector  informal  urbano
en  el  Per\'u  ha  sido un fen\'omeno caracter\'istico de las \'ultimas  dos
d\'ecadas,  debido  al acelerado proceso de migraci\'on y urbanizaci\'on que
sufrieron  muchas ciudades, la aparici\'on del autoempleo y de una gran cantidad
de  unidades  econ\'omicas  de  peque\~na  escala,  frente  a  las  limitadas  fuentes
de  empleo  asalariada y formal para el conjunto de integrantes de la PEA (Poblaci\'on
Economicamente Activa).

Claramente se denota la importancia de las MiPYME en la econom\'ia nacional debido a su aporte a la
producci\'on nacional que seg\'un \citep{arbulu} fue de 42\% el a\~no 2005 y por
su potencial de absorci\'on de empleo que fue cerca del 88\%.

\cite{arbulu} agrega que tal es la importancia de las MiPYMES que \'estas aportar\'on
49\% al PBI el a\~no 2005, mientras el que \citep{produce} indica que el aporte
de las MiPYMES al PBI fue de 40\% el a\~no 2013 lo que significa un descenso significativo,
sin embargo ambos casos porcentajes no dejan de ser importantes.

Respecto a las MiPYMES y el empleo \cite{produce2} muestra que 10 de cada 100 personas
de la Poblaci\'on Economicamente Activa (PEA) son conductoras de una MiPYME formal.
Adem\'as agrega que \'estas gener\'an el 60\% de la PEA ocupada. El 83,5\% de las
Mipyme formales tienen hasta cinco trabajadores, esta estructura no presenta cambios
sustanciales respecto a a\~nos anteriores y es explicado por el gran n\'umero de
Mipyme que se constituyen como personas naturales y presentan bajos niveles de
ventas. Sin embargo, esta proporci\'on var\'ia de acuerdo con el estrato empresarial:
el 86,7\% de las microempresas, 14,4\% de las peque\~nas empresas y 4,3\% de las
medianas empresas tienen como m\'aximo cinco trabajadores.

Pese a lo indicado en los parrafos anteriores, uno de los problemas que aqueja a
las mipymes es la informalidad. \cite{loayza} precisa que la informalidad est\'a
constituido por el conjunto de empresas, trabajadores y actividades que operan
fuera de los marcos legales y normativos que rigen la actividad econ\'omica. Por
lo tanto, pertenecer al sector informal supone estar al margen de las cargas
tributarias y normas legales, pero tambi\'en implica no  contar  con  la  protecci\'on
y los servicios que el Estado  puede  ofrecer.

\cite{loayza} se\~nala desde un punto de vista conceptual que la informalidad surge
cuando los costos de circunscribirse al marco legal y normativo de un pa\'is son
superiores a los beneficios que ello conlleva. La formalidad involucra costos
tanto en t\'erminos de ingresar a este sector - largos,complejos y costosos procesos
de inscripci\'on y registro - como en t\'erminos de permanecer dentro del mismo -pago
de impuestos, cumplir las normas referidas a beneficios laborales
y remuneraciones, manejo ambiental, salud, entre otros. A lo que \citep{penaranda}
agrega que existen una diversidad de enfoques que explican la presencia de la
informalidad en un sistema econ\'omico, como las barreras burocr\'aticas y los sobrecostos
laborales y tributarios, as\'i como tambi\'en los distintos mecanismos a trav\'es de
los cuales esta afecta la productividad y el potencial de crecimiento de una
econom\'ia. En el accionar de una econom\'ia informal se encuentran empresas que
buscan eludir el control del Estado, manteniendo un tama\~no inferior al \'optimo
para gozar de beneficios tributarios o laborales, empleando mecanismos irregulares
para la compra de bienes y servicios e incluso destinando recursos financieros
para encubrir actividades ilegales.

Respecto al n\'umero de mipymes informales \cite{produce2} se\~nala que existe una
amplia heterogeneidad en las dimensiones de la informalidad, lo que dificulta
encontrar el n\'umero de empresas formales e informales de una manera \'unica y precisa.
Al respecto \citep{produce2} cita a (Diaz, 2014) que sintetiza en dos dimensiones
la informalidad: informalidad laboral e informalidad tributaria. En el caso de la
primera se distingue varios criterios (rasgos) para identificar las obligaciones
propias de una relaci\'on laboral como el acceso a un seguro de salud, una pensi\'on
de jubilaci\'on, gratificaciones y a un contrato de trabajo. En el caso de la segunda,
tambi\'an distingue criterios, como la tenencia de RUC de la empresa, si esta tiene
un sistema de contabilidad, y si se encuentra registrada como persona jur\'idica.

Respecto a la informalidad \cite{inei2} indica que el mayor n\'umero de unidades
productivas informales se concentra en la actividad Agropecuaria (33,8\%), le
siguen las actividades de Comercio (23,9\%), Transportes (12,2\%) y Otros servicios (10,9\%).

La tabla \ref{tabla_informalidad} muestra una estimaci\'on del n\'umero de micro y
peque\~nas empresas informales entre los a\~nos 2010 y 2014 recogido por
\citep{produce2}.

\begin{table}[htbp]
    \caption{Estimaci\'on del n\'umero de micro y peque\~nas empresas informales,
             2010-2014}
    \label{tabla_informalidad}
    \centering
        \begin{tabular}{|p{1.0cm}|p{3cm}|p{3cm}|p{3cm}|p{2cm}|}
            \hline
                \thead{Tama\~no} &
                \thead{N\'umero total \\estimado de micro y \\peque\~nas empresas} &
                \thead{Micro y peque\~nas \\empresas Formales} &
                \thead{Micro y peque\~nas \\empresas informales} &
                \thead{En porcentajes \\(\% f/i)} \\ \hline
            % Primera fila
            2010 &
            3,939,773 &
            1,199,347 &
            2,740,426 &
            30.4 ; 69.6 \\ \hline

            2011 &
            3,858,975 &
            1,289,107 &
            2,569,868 &
            33.4 ; 66.6 \\ \hline

            2012 &
            3,842,114 &
            1,345,390 &
            2,496,724 &
            35.0 ; 65.0 \\ \hline

            2013 &
            3,658,808 &
            1,518,469 &
            2,140,339 &
            41.5; 58.5 \\ \hline

            2014 &
            3,637,720 &
            1,597,061 &
            2,040,659 &
            43.9 ; 56.1 \\ \hline
        \end{tabular}
        \thead{Elaboraci\'on: PRODUCE - Direcci\'on de Estudios Econ\'omicos de Mype e Industria (DEMI)I}
\end{table}

\cite{perucamaras} expone un an\'alisis sobre la informalidad en la Macro Regi\'on
Sur de nuestro pa\'is donde el 79,1\% de la poblaci\'on ocupada labora en la informalidad,
mientras que solo el 20,9\% son trabajadores formales, agregando adem\'as que
2'013,914 personas en esta parte del pa\'is cuentan con empleos que no estan sujetos
a la legislaci\'on laboral nacional o que no pertenecen al sector formal de la
economia. Respecto a la actividad economica, se observa que en los sectores agropecuario
y pesquero la tasa de informalidad es de 99,2\%. Estas actividades concentran al
32,2\% de la Poblaci\'on Economicamente Activa (PEA) ocupada de esta macro regi\'on.



Es indudable que las Mipymes se han convertido en los actores primordiales en la
econom\'ia peruana, pero es necesario indicar que a\'un no se encuentran listas
para participar en el comercio internacional pues sus niveles tecnol\'ogicos no
les permiten adaptarse a los nuevos flujos de informaci\'on.

A pesar de las Uno de los principales problemas de las Mypime peruanas es la informalidad.

%---------------------------------------------------------------------
%                       Formulaci\'on del problema
%---------------------------------------------------------------------
\section{Formulaci'on del problema}
\subsection{Problema General}
?`Cual es el efecto del uso de las herramientas de la computaci\'on en la nube
en la gesti'on de las MiPYME pertenecientes al Centro de Desarrollo Empresarial
del Cusco?
\subsection{Problemas Espec'ificos}
\begin{enumerate}[a.]
\item ?`Cu\'al es el nivel de gesti\'on de las MiPYMES pertenecientes al Centro
de Desarrollo Empresarial del Cusco antes del uso de herramientas de la computaci\'on
en la nube?
\item ?`Cu\'al es el nivel de gesti\'on de las MiPYMES pertenecientes al Centro
de Desarrollo Empresarial del Cusco despu\'es del uso de herramientas de la computaci\'on
en la nube?
\end{enumerate}
%
%\subsubsection{Alcance Descriptivo}
%\subsubsection{Alcance Correlacional}
%\subsubsection{Alcance Explicativo}
%
%---------------------------------------------------------------------
%                          Justificaci\'on
%---------------------------------------------------------------------
\section{Justificaci'on}
\subsection{Conveniencia}

\subsection{Relevancia social}
%Las MiPYMES, como ya se menciono en las secciones anteriores de este cap\'itulo,
%tienen problemas en el acceso a las tecnolog\'ias de informaci\'on por diversos motivos
%que retrasa su crecimiento. Al realizar este estudio

%\subsection{Implicancias pr'acticas}
%\subsection{Valor teorico}
%\subsection{Utilidad metodol'ogica}

%---------------------------------------------------------------------
%                          Objetivos
%---------------------------------------------------------------------
\section{Objetivos de Investigaci'on}
\subsection{Objetivo General}
Determinar el efecto del uso de la computaci\'on en la nube en la gesti\'on de
las micro, peque\~nas y medianas empresas pertenecientes al Centro de desarrollo
empresarial del Cusco.
\subsection{Objetivos Especificos}
\begin{enumerate}[a.]
\item Determinar el nivel de gesti\'on de las MiPYMES pertenecientes al Centro de
Desarrollo Empresarial del Cusco antes del uso de las herramientas de la
computaci\'on en la nube.
\item Determinar el nivel de gesti\'on de las MiPYMES pertenecientes al Centro de
Desarrollo Empresarial del Cusco despu\'es del uso de las herramientas de la
computaci\'on en la nube.
\end{enumerate}
%---------------------------------------------------------------------
%                          Delimitaci\'on
%---------------------------------------------------------------------
\section{Delimitaci'on del estudio}
\subsection{Delimitaci'on espacial}
% Colocar una descripcion de la CDE en esta parte
Este trabajo de investigaci\'on se realizo en las MiPYMES pertenencientes al
Centro de Desarrollo Empresarial, la cual agrupa a micro, peque\~nas y
medianas empresas de distintos rubros de la ciudad del Cusco.

\subsection{Delimitaci'on temporal}
Este trabajo se desarrollar\'a en el periodo comprendido entre el .....
