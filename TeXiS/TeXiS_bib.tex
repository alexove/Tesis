%---------------------------------------------------------------------
%
%                         TeXiS_bib.tex
%
%---------------------------------------------------------------------
%
% TeXiS_bib.tex
% Copyright 2009 Marco Antonio Gomez-Martin, Pedro Pablo Gomez-Martin
%
% This file belongs to TeXiS, a LaTeX template for writting
% Thesis and other documents. The complete last TeXiS package can
% be obtained from http://gaia.fdi.ucm.es/projects/texis/
%
% This work may be distributed and/or modified under the
% conditions of the LaTeX Project Public License, either version 1.3
% of this license or (at your option) any later version.
% The latest version of this license is in
%   http://www.latex-project.org/lppl.txt
% and version 1.3 or later is part of all distributions of LaTeX
% version 2005/12/01 or later.
%
% This work has the LPPL maintenance status `maintained'.
%
% The Current Maintainers of this work are Marco Antonio Gomez-Martin
% and Pedro Pablo Gomez-Martin
%
%---------------------------------------------------------------------
%
% Fichero que contiene la generaciXn de la bibliografXa. En principio,
% no harXa falta tenerlo en  un fichero separado, pero como permitimos
% aXadir una frase cXlebre antes  de la primera cita, la configuraciXn
% ya  no es  trivial.  Para  "ocultar" la  fontanerXa  de LaTeX,  estX
% separada  la  configuraciXn  de  los  parXmetros  de  la  generaciXn
% concreta de la bibliografXa. La configuraciXn estX en el "directorio
% del usuario" (Cascaras), mientras  que la generaciXn se encuentra en
% el directorio TeXiS (este fichero).
%
%---------------------------------------------------------------------

%%%
% GestiXn de la configuraciXn
%%%

% Ficheros .bib
\def\ficherosBibliografia{}
\newcommand{\setBibFiles}[1]{
\def\ficherosBibliografia{#1}
}

% Frase cXlebre
\def\citaBibliografia{}
\newcommand{\setCitaBibliografia}[1]{
\def\citaBibliografia{#1}
}

%%%
% ConfiguraciXn terminada
%%%

%%%
%% COMANDO PARA CREAR LA BIBLIOGRAFXA.
%% CONTIENE TODO EL CXDIGO LaTeX
%%%
\newcommand{\makeBib}{

%
% Queremos que  tras el tXtulo del  capXtulo ("BibliografXa") aparezca
% una frase cXlebre,  igual que en el resto  de capXtulos. El problema
% es que  aquX no ponemos  nosotros a mano  el \chapter{BibliografXa},
% sino que lo mete Xl automXticamente.
%
% Afortunadamente,  la gente  de  bibtex  hace las  cosas  bien ;-)  y
% despuXs  de insertar  el tXtulo  de  la secciXn  ejecuta un  comando
% denominado  \bibpreamble  que por  defecto  no  hace  nada. Pero  si
% sobreescribimos  ese comando,  podremos  ejecutar cXdigo  arbitrario
% justo  despuXs de la  inserciXn del  tXtulo, y  antes de  la primera
% referencia. Por  tanto, lo que hacemos es  sobreescribir ese comando
% (normalmente  se conocen  como "hooks")  para aXadir  la  cita justo
% despuXs del tXtulo.
%
% Desgraciadamente,  dependiendo  de la  versiXn  de  Natbib, hay  que
% definir  o redefinir  el comando  (es decir,  utilizar  newcommand o
% renewcommand)... como eso es un  lXo, utilizamos let y def, pues def
% no falla si ya estaba definido.

\let\oldbibpreamble\bibpreamble

\def\bibpreamble{%
\oldbibpreamble
% AXadimos a la tabla de contenidos la bibliografXa. Si no lo hacemos
% aquX, sale mal o el nXmero de pXgina (grave) o el enlace en el PDF
% que te lleva a un sitio cercano (no tan grave).
% Para que la bibliografXa no salga en el PDF aXadida a la Xltima
% parte del documento, si hay partes en el mismo, le damos la
% categorXa de "parte".
\ifx\generatoc\undefined
\else
\ifx\tienePartesTeXiS\undefined
\addcontentsline{toc}{chapter}{Bibliografía}
\else
\addcontentsline{toc}{part}{Bibliografía}
\fi
\fi
% AXadimos tambiXn una etiqueta, para poder referenciar el nXmero
% de pXgina en el que comienza
\label{bibliografia}
% Frase cXlebre configurada por el usuario
\citaBibliografia
}
% Fin definiciXn "bibpreamble"


%
% Cambiamos  el estilo  de la  cabecera.  Hay que  hacerlo porque  por
% defecto el paquete que  estamos usando (fancyhdr) pone en mayXsculas
% el tXtulo completo del capXtulo.  Con los capXtulos normales esto se
% pudo evitar  en el  preXmbulo redefiniendo el  comando \chaptermark,
% pero con la bibliografXa no se puede hacer. Se define la cabera para
% que aparezca la palabra "BibliografXa" en ambas pXginas.
%
%
\cabeceraEspecial{Bibliografía}


% Creamos la  bibliografXa. Lo hacemos  dentro de un bloque  (entre la
% pareja \begingroup  ... \endgroup) porque  dentro vamos a  anular la
% semXntica que da babel  a la tilde de la eXe (que  hace que un ~N se
% convierta  automXticamente en una  X). Esto  es debido  a que  en el
% bibtex  aparecerXn  ~ para  separar  iniciales  de  los nombres  con
% espacios  no separables  en varias  lineas, y  aquellos  nombres que
% tengan  una N  como  inicial serXan  puestos  como X.  Al anular  la
% semXntica     al    ~    que     da    babel,     deshacemos    este
% comportamiento. Naturalmente,  para que esto  no tenga repercusiones
% negativas,  en  ningXn  .bib  deberXamos  utilizar ~N  (o  ~n)  para
% representar una X ... tendremos  que utilizar o una X/X directamente
% (no  aconsejable porque asume  que hemos  usado inputenc)  o, mejor,
% usamos la  versiXn larga  \~n o  \~N que no  falla nunca.   Para mXs
% informaciXn,  consulta  el  TeXiS_pream.tex  en el  punto  donde  se
% incluye natbib y babel.

\begingroup
\spanishdeactivate{~}
\bibliographystyle{TeXiS/TeXiS}
%\bibliographystyle{abbrv}
\bibliography{\ficherosBibliografia}
\endgroup

} %\newcommand{\makeBib}

% Variable local para emacs, para  que encuentre el fichero maestro de
% compilaciXn y funcionen mejor algunas teclas rXpidas de AucTeX

%%%
%%% Local Variables:
%%% mode: latex
%%% TeX-master: "../Tesis.tex"
%%% End:
