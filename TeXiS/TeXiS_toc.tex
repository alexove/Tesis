% Pedimos que inserte todos los epXgrafes hasta el nivel \subsection en
% la tabla de contenidos.
\setcounter{tocdepth}{2}

% Le  pedimos  que nos  numere  todos  los  epXgrafes hasta  el  nivel
% \subsubsection en el cuerpo del documento.
\setcounter{secnumdepth}{3}


% Creamos los diferentes Xndices.

% Lo primero un  poco de trabajo en los marcadores  del PDF. No quiero
% que  salga una  entrada  por cada  Xndice  a nivel  0...  si no  que
% aparezca un marcador "Xndices", que  tenga dentro los otros tipos de
% Xndices.  Total, que creamos el marcador "Xndices".
% Antes de  la creaciXn  de los Xndices,  se aXaden los  marcadores de
% nivel 1.

\ifpdf
   \pdfbookmark{\'Indices}{indices}
\fi

% Tabla de contenidos.
%
% La  inclusiXn  de '\tableofcontents'  significa  que  en la  primera
% pasada  de  LaTeX  se  crea   un  fichero  con  extensiXn  .toc  con
% informaciXn sobre la tabla de contenidos (es conceptualmente similar
% al  .bbl de  BibTeX, creo).  En la  segunda ejecuciXn  de  LaTeX ese
% documento se utiliza para  generar la verdadera pXgina de contenidos
% usando la  informaciXn sobre los  capXtulos y demXs guardadas  en el
% .toc
\ifpdf
   \pdfbookmark[1]{Tabla de contenidos}{tabla de contenidos}
\fi

\cabeceraEspecial{\'Indice}

\tableofcontents

\newpage

% Xndice de figuras
%
% La idea es semejante que para  el .toc del Xndice, pero ahora se usa
% extensiXn .lof (List Of Figures) con la informaciXn de las figuras.

\cabeceraEspecial{\'Indice de figuras}

\ifpdf
   \pdfbookmark[1]{\'Indice de figuras}{indice de figuras}
\fi

\listoffigures

%\newpage

% Xndice de tablas
% Como antes, pero ahora .lot (List Of Tables)

\ifpdf
   \pdfbookmark[1]{\'Indice de tablas}{indice de tablas}
\fi

\cabeceraEspecial{\'Indice de tablas}

\listoftables

%\newpage

% Variable local para emacs, para  que encuentre el fichero maestro de
% compilaciXn y funcionen mejor algunas teclas rXpidas de AucTeX

%%%
%%% Local Variables:
%%% mode: latex
%%% TeX-master: "../Tesis.tex"
%%% End:
