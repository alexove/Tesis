%---------------------------------------------------------------------
%
% Fichero que  contiene la portada y  primera hoja de la  tesis, en la
% que se vuelve a repetir el tXtulo.
%
% El contenido  de la  portada se configura  en un fichero  externo en
% Cascaras/cover.tex
%
%---------------------------------------------------------------------

%%%
% GestiXn de la configuraciXn
%%%

% InstituciXn
\def\institucionVal{}
\newcommand{\institucion}[1]{
\def\institucionVal{#1}
}

% TXtulo de la portada
\def\tituloPortadaVal{\titulo}
\newcommand{\tituloPortada}[1]{
\def\tituloPortadaVal{#1}
}

% Autor
\def\autorPortadaVal{\autor}
\newcommand{\autorPortada}[1]{
\def\autorPortadaVal{#1}
}

% Tutor
\def\directorPortadaVal{Tutor no definido. Usa
  \texttt{$\backslash$tutorPortada}}
\newcommand{\directorPortada}[1]{
\def\directorPortadaVal{#1}
}

% Fecha de publicacion
\def\fechaPublicacionVal{}
\newcommand{\fechaPublicacion}[1]{
\def\fechaPublicacionVal{#1}
}

% Imagen de la portada (escudo)
\def\imagenPortadaVal{Imagenes/Vectorial/Todo}
\newcommand{\imagenPortada}[1]{
\def\imagenPortadaVal{#1}
}
\def\escalaImagenPortadaVal{1.0}
\newcommand{\escalaImagenPortada}[1]{
\def\escalaImagenPortadaVal{#1}
}
% Tipo de documento (TESIS, MANUAL, ...)
\def\tipoDocumentoVal{TESIS DOCTORAL}
\newcommand{\tipoDocumento}[1]{
\def\tipoDocumentoVal{#1}
}



% Primer subtXtulo de la segunda portada
%\def\textoPrimerSubtituloPortadaVal{%
%\textit{Memoria que presenta para optar al tXtulo de Doctor en InformXtica}  \\ [0.3em]%
%\textbf{\autorPortadaVal}%
%}
%\newcommand{\textoPrimerSubtituloPortada}[1]{
%\def\textoPrimerSubtituloPortadaVal{#1}
%}

% Segundo subtXtulo de la segunda portada
%\def\textoSegundoSubtituloPortadaVal{%
%\textit{Dirigida por el Doctor}  \\ [0.3em]
%\textbf{\directorPortadaVal}
%}
%\newcommand{\textoSegundoSubtituloPortada}[1]{
%\def\textoSegundoSubtituloPortadaVal{#1}
%}

% ISBN
%\newcommand{\isbn}[1]{
%\def\isbnVal{#1}
%}

% Copyright
%\newcommand{\copyrightInfo}[1]{
%\def\copyrightInfoVal{#1}
%}

% CrXditos a TeXiS
%\newcommand{\noTeXiSCredits}{
%\def\noTeXiSCreditsVal{}
%}

% ExplicaciXn sobre impresiXn a doble cara
%\newcommand{\explicacionDobleCara}{
%\def\explicacionDobleCaraVal{}
%}
%%%
% ConfiguraciXn terminada
%%%


%%%
%% COMANDO PARA CREAR LAS PORTADAS.
%% CONTIENE TODO EL CXDIGO LaTeX
%%%
\newcommand{\makeCover}{


% Ponemos el marcador en el PDF
\ifpdf
\pdfbookmark{Car\'atula}{titulo}
\fi

%
%  MXRGENES
%
% La maquetaciXn de las pXginas en LaTeX es bastante complicada en lo
% que se refiere a los mXrgenes. Por ejemplo, _siempre_ hay un
% desplazamiento de una pulgada hacia la derecha y hacia abajo, porque
% por razones de maquetaciXn se cree que es necesario ese espacio. Si
% quieres (en una hoja impar, que son de las Xnicas que me he
% preocupado) escribir por encima de la primera pulgada, o hacia la
% izquierda, tienes que usar "mXrgenes" negativos.
%
% Para poder seguir esto un poco, lo mejor es que ejecutes \layout con
% el paquete layout cargado, para ver una imagen...
%
% Los valores importantes en lo que se refiere al margen (horizontal,
% que es del Xnico que me he preocupado) son:
%
%   - \hoffset : desplazamiento horizontal del "eje" de coordenadas. A
%   este valor hay que sumarle, irremediablemente, una pulgada. El
%   valor por defecto es 0 pt
%   - \oddsidemargin : "margen" izquierdo (en las pXginas impares). El
%   texto principal de los pXrrafos comenzarX en esa posiciXn (es
%   decir, 1 pulgada + \hoffset + \oddsidemargin). El valor por
%   defecto es 22 pt
%   - \textwidth : longitud del texto (de los pXrrafos). El valor por
%   defecto son 360 pt.
%   - \marginparsep : separador de la parte derecha del texto
%   principal y el espacio a anotaciones en el margen (notas
%   marginales). El valor por defecto es 7 pt.
%   - \marginparwidth : ancho de la secciXn de notas marginales. El
%   valor por defecto es 106 pt.
%
% Fijate que las notas marginales NO necesariamente llegarXn hasta el
% final del folio. La separaciXn entre el extremo derecho de las notas
% al margen y el final del folio en realidad dependerX del tamaXo de
% Xste; no se especifica de ninguna manera. Esto significa que si
% quieres ajustar exactamente el margen derecho tienes que echar
% cuentas con respecto al tamaXo del papel, que se mantiene en
% \paperwidth.
%
% En realidad, no sX de quX manera pero todos los valores estXn
% relacionados de algXn modo, y un cambio en uno afecta a los demXs de
% maneras bastante inverosXmiles. AdemXs, _sXlo_ pueden cambiarse en
% el preXmbulo (bueno... al menos \textwidth, \oddsidemargin se puede
% cambiar en otros sitios, pero no se pueden hacer muchas cosas sXlo
% con aquellos que se pueden cambiar...).
%
% En general se recomienda que NO cambies los mXrgenes. EstXn elegidos
% por especialistas que saben de maquetaciXn y que han estudiado en
% profundidad las mejores organizaciones. Por ejemplo, los cambios que
% hagas pueden alargar demasiado las lXneas, o dejarlas demasiado
% cortas. Pero bueno, si aun asX quieres cambiar los mXrgenes _a nivel
% global_ es preferible que uses el paquete geometry, cuya inclusiXn
% recibe los centXmetros de mXrgen que quieres en cada lado, y Xl se
% encarga de hacer las cuentas para que queden asX, porque tocar tanto
% valor es un infierno.
%
% Si quieres cambiar los mXrgenes momentXneamente, entonces lo mejor
% es hacer uso de un entorno tipo "lista" que permite tocar algunos
% contadores para ajustar las posiciones de los pXrrafos. Eso es
% precisamente lo que hace el entorno cambiamargen definido en
% TeXiS.sty
%
% El problema es que esos cambios son _relativos_ a los mXrgenes
% oficiales. Si quieres hacer un cambio drXstico (como el que
% necesitamos en la portada, para que quede centrada), entonces es
% necesario echar cuentas con los contadores anteriores para realizar
% el desplazamiento adecuado en cada lado para conseguirlo.
%
% Para resumir, los valores usados en los mXrgenes eran:
% 1 pulgada + hoffset + oddsidemargin +
%          + textwidth +
% + marginparsep + marginparwidth + AJUSTE
%                                          = paperwidth
%
% Lo que necesitamos en crear un entorno cambiamargen pasando los
% valores adecuados para que quede centrado. Para eso, hay que hacer
% cuentas, y eso es un tanto infierno en LaTeX a no ser que se incluya
% el paquete calc que permite notaciXn infija de operadores. Por
% tanto, para que esto funcione hay que incluirlo.
%
% Para aclararnos, vamos a crear un par de longitudes para hacer las
% cuentas poco a poco. AdemXs, lo primero es asumir que queremos
% _eliminar todos los mXrgenes_ y que los pXrrafos lleguen totalmente
% de lado a lado. Para eso, en la izquierda tendremos que restar la
% suma de los tres primeros valores (1 pulgada, \hoffset y
% \oddsidemargin).

\newlength{\cambioIzquierdo}
\setlength{\cambioIzquierdo}{1in + \hoffset + \oddsidemargin}
% Si te falla aquX, incluye el paquete calc en el preXmbulo.
% 1in = 1 pulgada = 72.27 pt

% En la parte derecha hay que restar el "margen" visible total, que es
% el tamaXo de la hoja restando el espacio hasta la izquierda del
% pXrrafo y su ancho. Aprovechamos que el primer espacio lo tenemos en
% \cambioIzquierdo.
\newlength{\cambioDerecho}
\setlength{\cambioDerecho}{\cambioIzquierdo + \textwidth}
\setlength{\cambioDerecho}{\paperwidth - \cambioDerecho}

% Ya casi estX. Si hicieramos
%
% \begin{cambiamargen}{-\cambioIzquierdo}{-\cambioDerecho}
%    ...
% \end{cambiamargen}
%
% tendriamos pXrrafos que van de extremo a extremo de la hoja. Como
% eso es una exageraciXn, restamos a cada longitud el mXrgen real que
% queremos dejar.
\newlength{\margenPortada}
\setlength{\margenPortada}{2.0cm}

\setlength{\cambioIzquierdo}{\cambioIzquierdo - \margenPortada}
\setlength{\cambioDerecho}{\cambioDerecho - \margenPortada}

%%%
% Portada
%%%


% PXgina sin cabeceras
\thispagestyle{empty}

\begin{cambiamargen}{-\cambioIzquierdo}{-\cambioDerecho}


% En la primera hoja no se entiende de pXginas pares e impares
\newlength{\evensidemarginOriginal}
\setlength{\evensidemarginOriginal}{\evensidemargin}

\newlength{\oddsidemarginOriginal}
\setlength{\oddsidemarginOriginal}{\oddsidemargin}

\setlength{\evensidemargin}{0cm}
\setlength{\oddsidemargin}{0cm}

% Comienza el tex...
% -- Institucion
\begin{huge}
	\begin{center}
		\textbf{\institucionVal}\\
	\end{center}
\end{huge}

\begin{center}
\rule{10.0cm}{.3mm}\\
%\vfill
\includegraphics[scale=\escalaImagenPortadaVal]{\imagenPortadaVal}
\end{center}
% // Institucion

% -- Titulo del tema
\begin{huge}
%\vfill
% Cuidado: Ajustarlo si el tXtulo cambia...
\newlength{\longTitulo}
\settowidth{\longTitulo}{%
\textbf{Arquitectura y metodologXa para el}}
\begin{center}
	\rule{18cm}{.3mm}\\
	%\vskip 0.3cm
	\textbf{\tituloPortadaVal}
%	\vskip 0.3cm
	\rule{18cm}{.3mm}\\
\end{center}
\end{huge}

%\vfill
% // Titulo del tema

%\begin{center}
%  {\Large \textbf{\tipoDocumentoVal}}
%\end{center}

%\vfill
%-----
%\makebox[10pt][r]{
%	\textbf{TESIS PARA OPTAR EL GRADO ACAD\'EMICO DE MAESTRO EN ADMINISTRACI\'ON DE NEGOCIOS\\
%	Presentado por: \\
%	Ing. \autorPortadaVal}\\[0.3cm]
%	ASESOR:
%}
%---
\begin{center}
\begin{tabular}[h]{m{3.5cm}m{8cm}}
& TESIS PARA OPTAR EL GRADO ACAD\'EMICO DE MAESTRO EN ADMINISTRACI\'ON DE NEGOCIOS \vfill \\
& Presentado por: \vfill \\
& Ing. \autorPortadaVal \vfill \\
& ASESOR: \directorPortadaVal \vfill \\
\end{tabular}
\end{center}
%---

%\begin{flushright}
%	\textbf{TESIS PARA OPTAR EL GRADO ACAD\'EMICO DE MAESTRO EN ADMINISTRACI\'ON DE NEGOCIOS\\
%	Presentado por: \\
%	Ing. \autorPortadaVal}\\[0.3cm]
%	ASESOR:
%\end{flushright}
%------

\end{cambiamargen}

% -- AXo de publicacion
\begin{center}
	\textbf{CUSCO - PER\'U\\
	\fechaPublicacionVal}
\end{center}
% // Anho de publicacion
%\newpage

% PXgina que en la cara de detrXs del folio de la portada.
% Ponemos que el documento esta maquetado con TeXiS y la
% aclaraciXn de que debe imprimirse a doble cara.
%\thispagestyle{empty}
%\mbox{ }
%\vfill%space*{4cm}
%\begin{small}
%\begin{center}
%\ifx\noTeXiSCreditsVal\undefined
%  Documento maquetado con \texis\ v.\texisVer.
%\else
%\mbox{ }
%\fi
%\end{center}
%\end{small}
%\vspace*{2cm}
%\begin{small}
%\begin{center}
%\ifx\explicacionDobleCaraVal\undefined
%\mbox{ }
%\else
%\noindent Este documento estX preparado para ser imprimido a doble
%cara.
%\fi
%\end{center}
%\end{small}

%%%
% Segunda portada
%%%

%\newpage

%\thispagestyle{empty}

%\mbox{ }

%\begin{Huge}
%\begin{center}
%\tituloPortadaVal
%\end{center}
%\end{Huge}

%\vfill

%\begin{large}
%\begin{center}
%\textoPrimerSubtituloPortadaVal
%\\ \mbox{ } \\ \mbox{ } \\
%\textoSegundoSubtituloPortadaVal \\ [0.3em]
%\end{center}
%\end{large}

%\vfill

%\begin{large}
%\begin{center}
%\textbf{\institucionVal}\\[0.2em]
%    \mbox{ }  \\
%\textbf{\fechaPublicacionVal}
%\end{center}
%\end{large}


%\newpage
%\thispagestyle{empty}
%\mbox{ }

% InformaciXn del ISBN y copyright
%\vskip 13cm
%\ifx\copyrightInfoVal\undefined
%\mbox{ }
%\else
%Copyright \textcopyright\ \copyrightInfoVal
%\fi
%\vskip 3cm
%\ifx\isbnVal\undefined
%\mbox{ }
%\else
%ISBN \isbnVal
%\fi

} % \newcommand{\makeCover}
